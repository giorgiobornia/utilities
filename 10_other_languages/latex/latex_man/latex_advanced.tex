% Copyright (c) 1999 by T.P.Love. This document may be copied freely 
% for the purposes of education and non-commercial research.
% Cambridge University Engineering Department,  
% Cambridge CB2 1PZ, England.
% Use latex file.tex ;latex file.tex ; dvips file.dvi
% to print a copy. 
\documentclass[dvips]{article}
\usepackage{multicol,longtable,palatino,a4,graphics,color}
\usepackage{array,pifont,fancyhdr,html}

\setlongtables
\begin{document}
\def\xdt{$\cal X\!\!$\texttt{.desktop}}

\setlength{\parskip}{0.4em}

\title{Advanced \LaTeX}
\author{Tim Love}
\date{\today}
\maketitle

\begin{quotation}
This document follows on from the 
\htmladdnormallinkfoot{\textit{Word processing using \LaTeX}}{http://www-h.eng.cam.ac.uk/help/tpl/textprocessing/latex\_basic/latex\_basic.html} 
document. It describes the features of \LaTeX{} that people at CUED
are most likely to use. Further information is available from the 
\htmladdnormallinkfoot{LaTeX help page}{http://www-h.eng.cam.ac.uk/help/tpl/textprocessing/LaTeX\_intro.html}
and in the books available for loan from the operators in the 
\texttt{DPO}.

Comments and bug reports to Tim Love (\htmladdnormallink{tpl@eng.cam.ac.uk}{mailto:tpl@eng.cam.ac.uk}). 
\end{quotation}

\begin{table}[b]
Copyright \copyright 1999 by T.P.~Love.
This document may be copied freely for the purposes 
of education and non-commercial research.
Cambridge University Engineering Department,  
Cambridge CB2 1PZ, England.
\end{table}

\tableofcontents

\pagestyle{fancy}
\section{\LaTeX\ Concepts}
\subsection{Environments and commands}
\LaTeX\ is a macro-package for \TeX\ which has many preset \emph{environments} 
where much of the setting up that \TeX\ users have to do explicitly is done
for you. An environment has the form

\verb|\begin{|\textit{environment name}\verb|}|

.

\verb|\end{|\textit{environment name}\verb|}|


\LaTeX\ also has commands which affect the formatting of the document.
Their arguments are given in braces. For example, 
\begin{verbatim}
  \textit{This is much more important} than this.
\end{verbatim}
produces as output  \textit{This is much more important} than this.

The related \verb|\itshape| command doesn't take an argument. It affects 
all the subsequent text in the environment where it's used.

\LaTeX\ tries to enforce the idea that the visual appearance of
the document (use of fonts, indentation, etc) should derive from the 
logical structure of the document (\textit{i.e.} rather than manually
putting the section titles into bold, you should let the \verb|\section|
command do it). Resisting this philosophy can lead to
extra (usually unnecessary) work.

\LaTeX\  is expandable. Many macros can be loaded in to provide added
featues. You can also create your own commands and environments. Commands
can take arguments that modify their action 
\begin{itemize}
\item Some commands have a \texttt{*-form}, a variant on the standard command
that you get by adding a \texttt{*} to the command name.

\item Mandatory arguments are enclosed in \verb|{}| braces
\item Optional arguments are enclosed in \verb|[ ]| brackets.  
\end{itemize}

\subsection{Classes and packages}
At the top of your file you will have a line something like
\begin{verbatim}
\documentclass[12pt]{article}
\end{verbatim}
which determines the font size and document 
class -- the type of document you're writing. 
Each class has an associated \verb|*.cls| file in the system directories
which is read in at 
start-up time.
Other options accepted
by \texttt{article} include \texttt{10pt}, \texttt{11pt} and 
\texttt{twocolumn}. 

Then you'll probably need to load in extra macros with the 
\verb|\usepackage| command. Each package has  an associated 
\verb|*.sty| file in the system directories. The packages 
inherit the options from the \verb|\documentclass| line and can be
given others of their own. \emph{E.g.}
\begin{verbatim}
\usepackage[dvips]{graphicx}
\usepackage[dvips]{color}
\end{verbatim}
tells \LaTeX{} that you want to use the extra \texttt{graphicx} and
\texttt{color} macros, and that you are going to use  \texttt{dvips}
to convert the resulting file to postscript. You can shorten this to
\begin{verbatim}
\usepackage[dvips]{graphicx,color}
\end{verbatim}

\subsection{Errors}

Errors can be reported either from the \LaTeX\ phase (in which case it is 
reported as such) or the lower level \TeX\ phase. Don't be too put off by the 
half-digested text displayed. A line number is reported which 
usually helps to detect the error. Typing \texttt{`h'} sometimes produces 
helpful diagnostics.   
The most common errors reported are
\begin{itemize}
\item a command misspelt

\item a mismatched brace

\item improper use of special characters

\item forgetting to have the appropriate \verb|\usepackage{...}| lines.

\item a error like this
\begin{small}
\begin{verbatim}
Overfull \hbox (15.42563pt too wide) in paragraph at lines 
285--288 \OT1/ppl/m/n/10 You can cre-ate ver-tical space 
between lines or ho-ri-zontal space   between
\end{verbatim}
\end{small}
means that lines 285 to 288 are producing a text line about 15 pts (about
5mm) too wide. \verb|\OT1/ppl/m/n/10| is the specification of the font
used.
When \TeX\ does right and left alignment, it works out how much space it 
needs to leave between words and where to hypenate words if necessary. But the
amount of space it's prepared to leave has to fall within a certain range
and it will only split words in certain places (shown in the error message  
by a hyphen). If these restrictions mean that \LaTeX\ can't produce a
satisfactory line, it will produce as much of the line as it can. A re-phrasing of
the offending sentence will usually solve the problem. Another thing you
might try is to control the way a troublesome word is hyphenated using
something like 
\begin{verbatim}
\hypenation{furthermore fur-ther-more}
\end{verbatim}
at the top of your file. If all else fails,  use 
\verb|{\sloppy...}| to enclose the offending text.
 

\item \LaTeX\ issues a \verb|*|.

 This means \LaTeX\ needs more input. It
probably means you've missed out an \verb|\end{document}|, but if not you may
be able to get \LaTeX\ to continue processing as best it can by  typing
\verb|<Return>|.

\item \verb|! Argument of \label  has an extra }|

If you're sure you haven't left out a left brace, then maybe you need
to \emph{protect} the inmost (fragile) command. For example, in some
older versions of \LaTeX{} 
\verb|\label| is fragile which causes a problem in 
\verb|\caption{Picture\label{margin}}|, so the safer construction 
\verb|\caption{Picture\protect\label{margin}}|
has to be used.
\end{itemize} 

\subsection{Files created}
More than just the \texttt{.dvi} file may be produced. Don't worry about them --
except perhaps for the log file you won't need to look at them.
\begin{itemize}
\item \texttt{.aux} - cross references, etc
\item \texttt{.toc} -  created by \verb|\tableofcontents|
\item \texttt{.lof} -  created by  \verb|\listoffigures|
\item \texttt{.lot} -  created by  \verb|\listoftables|
\item \texttt{.log} -  a copy of the diagnostic output that usually comes out on the
screen.
\end{itemize}

Some common errors can be found using the \texttt{lacheck}
program on the \LaTeX{} file.

\subsection{How to use \LaTeX{} at CUED}
\begin{description}
\item[From the command line --] 
After running \texttt{latex} at least twice, you should be able to 
preview your \verb|*.dvi| document using \texttt{xdvi} and print it using 
\texttt{plotview}, but if postscript is implicated at all (if you load
in graphics, use postscript fonts, scale, use color, or rotate) then
you should convert your  \verb|*.dvi| document to postscript. A typical
sequence of commands to process \texttt{doc.tex} would be
\begin{verbatim} 
 latex doc
 latex doc
 dvips doc.dvi -o doc.ps
 ghostview doc.ps
 lp -dljmr1 -opostscript doc.ps 
\end{verbatim} 

\item[Using \texttt{xlatex} --] \texttt{xlatex} has buttons to process,
preview and print your document (or selected pages of it), and convert it to 
postscript. Just type \texttt{xlatex} \emph{filename}.
\end{description}

Writing a \LaTeX\ document is rather like writing a program. This makes
using \LaTeX\ more difficult in some respects than using a word 
processor, but there are advantages too. For instance creating a table
of contents is trivial. Beginners often use unnecessary
`\verb|\\|' sequences and write 
`\verb|{\large \textbf{2.1 Method}}\\|' when `\verb|\subsection{Method}|' 
would be much better. Users who think they know more about typesetting than
\LaTeX\ (those who, for example, like underlining) will waste a lot of
time too.

Avoid repeating constructions. Instead, write your own macros and commands, 
and familiarise yourself with the packages 
described in the 
\htmladdnormallinkfoot{packages}
{http://www-h.eng.cam.ac.uk/help/tpl/textprocessing/LaTeX\_intro.html\#Packages} section of the online \LaTeX{} page.

\section{Document structure}
\subsection{Counters and Length parameters}
\begin{itemize}
\item \textbf{Counters} :-
\LaTeX\ maintains many counter variables (\emph{e.g.} \texttt{page}, 
\texttt{part}, \texttt{equation}, 
\texttt{footnote}, \texttt{chapter}, \texttt{paragraph}, \texttt{section}, 
\texttt{subsection}, \texttt{subsubsection}, \texttt{enumi}, 
etc). You can set these counters yourself. Some
examples:-
\begin{verbatim}
\setcounter{page}{0}
\addtocounter{chapter}{2}
\end{verbatim}


\item \textbf{Length Parameters} :- 
\LaTeX\ accepts the following units of length: \texttt{in, cm,  mm, pt} (there
are 72.27 pts to an inch), \texttt{em} (width of an \texttt{M}),
\texttt{ex} (height of an \texttt{x}). These units can be used to set the values of length 
variables using \verb|\setlength|. For example,
\begin{verbatim}
\setlength{\parindent}{0in}
\end{verbatim}
sets to zero the amount by which the first line of a paragraph is indented.
Other useful length parameters are:-
\begin{description}
\item[\texttt{parskip}:-]   determines the gap between paragraphs.
\item[\texttt{baselineskip}:-] determines the usual distance from the bottom of one line to the bottom 
                     of the next. You can adjust this to produce double 
spacing, but a better way, which takes a stretch \emph{factor} as argument
is to use \verb|\linespread|. 
For instance \emph{before} \verb|\begin{document}| you can do
\begin{verbatim}
\linespread{1.6}
\end{verbatim}
to get doublespacing through your document. The \texttt{setspace} package
offers more options.
\end{description}
\end{itemize}

\subsection{Document and page organisation}
\begin{itemize}
\item \textbf{Document classes}: The standard classes are  \texttt{article}, 
\texttt{book}, \texttt{letter}, \texttt{report},
\texttt{slides}. 
The differences between these are often minor. A 
\texttt{book} can have chapters. In a \texttt{report} sections begin at \texttt{0}
whereas in an \texttt{article} they begin at \texttt{1}. Just about all our 
handouts are \texttt{article}s. 

\item \textbf{Big Documents} :- It's best to split your document into smaller files and have a master file
looking like this
\begin{verbatim}
 \documentclass[dvips,12pt]{book}
 \usepackage{a4,color,graphics,palatino,fancyhdr}
 \begin{document}
 \pagestyle{empty}
 \tableofcontents
 \listoffigures
 \pagestyle{fancy}
 \input{chapter1}
 \input{chapter2}
 \input{fig2}
 \input{chapter3}
 \input{chapter4}
 \appendix
 \input{appendices}
 \end{document}
\end{verbatim} 
on which you can run \texttt{latex} just as if the master file contained all the text of \texttt{chapter1.tex}
etc. The advantage of this is that once you have a chapter correct, you can comment out the 
corresponding `input' line and avoid unnecessary processing. Remember to
take out the \verb|\begin{document}| and \verb|\end{document}| lines from the 
component files.    


\item \textbf{Page Size} :- 
You can choose the margin sizes yourself by changing the following 
dimensions \emph{before} the \verb|\begin{document}| line. 
\begin{verbatim}
\setlength{\topmargin}{-0.4in}
\setlength{\topskip}{0.3in}    % between header and text
\setlength{\textheight}{9.5in} % height of main text
\setlength{\textwidth}{6in}    % width of text
\setlength{\oddsidemargin}{0.75in} % odd page left margin
\setlength{\evensidemargin}{0.75in} % even page left margin
\end{verbatim}

Note that a margin width of 0cm gives you a margin 4cm wide. Rather 
than set absolute sizes you can modify the default sizes using commands
like the following --

\begin{verbatim}
\addtolength{\evensidemargin}{-1cm}
\addtolength{\oddsidemargin}{-1cm}
\addtolength{\textwidth}{2cm}
\end{verbatim}
Further support for control of page layout is provided by the 
\htmladdnormallinkfoot{geometry}{http://www-h.eng.cam.ac.uk/help/tpl/textprocessing/geometry.dvi} package.
To see the current values of these dimensions, use the 
\verb|layout| package, which defines a \verb|\layout| command. 

For A3 output,  add \verb|\usepackage{a3}| to your 
  latex source file (\texttt{foo.tex} say), run latex on it, convert the 
resulting file to 
  Postscript using ``\verb|dvips -t a3 foo.dvi -ofoo.ps|'' then print using 
``\verb|lp -oa3 -dljmr2 foo.ps|''.




\item \textbf{Title Pages} :-
The title page of this document was created by the following \LaTeX\ commands.
\begin{verbatim}
\title{Advanced \LaTeX}
\author{Tim Love}
\date{\today}
\maketitle
\end{verbatim} 

\item \textbf{Contents} :- Use \verb|\tableofcontents| to create a contents list 
at that point in the document. \LaTeX\ will pick out the sections,
subsections etc for you. You'll have to run \LaTeX\ at least twice though.

\item \textbf{Page numbers and Headings} :- These are determined by the 
argument given in \verb|pagestyle{}|.

\begin{description}
\item[ \texttt{plain} :-]  the page numbers are put at the bottom of the page.
The top of the page is empty. This is the default mode.

\item[ \texttt{empty} :-] this suppresses page numbering altogether, except
on the title page it you're using \verb|\maketitle|. The workaround in
this case is to do
\begin{verbatim}
\maketitle
\thispagestyle{empty}
\end{verbatim}
with no gap between the two lines.


\item[\texttt{headings} :-] this puts the page numbers at the top 
and adds a header whose contents depend on the document style.
\end{description}

\htmladdnormallinkfoot{\texttt{fancyhdr}}{http://www-h.eng.cam.ac.uk/help/tpl/textprocessing/fancyhdr.dvi}
is a popular package that adds useful page
headers when the command \verb|\pagestyle{fancy}| is used. This handout uses it.

Long section titles can cause trouble in headers. The section commands
let you specify an extra, shorter title for use in the header and contents 
page. Section \ref{LONGTITLE} was specified as follows
\begin{verbatim}
\subsection[Pagebreaks, footnotes, etc]
   {Pagebreaks, space, footnotes, references, boxes, etc}
\end{verbatim}

\item \textbf{Sectioning} :- To start a chapter 
called \textbf{Life} in a \texttt{book}, just use \verb|\chapter{Life}|.
Similar commands to start a \texttt{part}, \texttt{section} or \texttt{subsection} 
also  exist in most document classes (articles don't have chapters or parts
though).

If you use the \texttt{*-form} of the command then the sections will not
be numbered, neither will it appear in the table of contents.

A title will only be numbered if its 'depth' isn't more than
\texttt{secnumdepth} and will only appear in the contents page if
the 'depth' isn't more than the value of \texttt{tocdepth}. So, for
example, doing
\begin{verbatim}
\setcounter{tocdepth}{2}
\setcounter{secnumdepth}{3}
\end{verbatim}
will cause section \texttt{1.4.3} to be numbered, but it won't appear in the
contents. 
\end{itemize}

\subsection[Pagebreaks, footnotes, etc]
   {Pagebreaks, space, footnotes, references, boxes, etc}
\label{LONGTITLE}
\begin{itemize}

\item \textbf{Page Breaks} :- you can force a page break using \verb|\newpage|.
\item \textbf{Preventing line breaks} :- If there's a word that you don't want broken, put it in an
\verb|mbox|. \emph{E.g.},
\begin{verbatim}
One shouldn't try to break up \mbox\emph{relationships} 
\end{verbatim}



\item \textbf{Space} :-
\begin{verbatim}
You can create vertical space \vspace{.5cm} between lines 
or horizontal space \hspace{1.5cm} between words. 
The arguments to these commands can be negative. 
\end{verbatim}
You can create vertical space \vspace{.5cm} between lines 
or horizontal space \hspace{1.5cm} between words. 
The arguments to these commands can be negative. \verb|\vspace*|
will create space even at the top of a page. It's sometimes useful
to create stretchable space. The following creates space that pushes
the letters to the edge of the page
\begin{verbatim}
\noindent A\hspace{\stretch{1}} B\\
C\hspace{\stretch{1}} D
\end{verbatim}

\noindent A\hspace{\stretch{1}} B\\
C\hspace{\stretch{1}} D


\item \textbf{Footnotes} :- 
\begin{verbatim}
Do them like this\footnote{I told you so.}
\end{verbatim}
Do them like this\footnote{I told you so.}
The footnotes are numbered by default. If you want to use symbols
(stars, daggers etc) then you need to redefine how the \texttt{footnote}
counter is displayed.
\begin{verbatim}
\def\thefootnote{\fnsymbol{footnote}}
\end{verbatim}

\item \textbf{Margin notes} :- 
\begin{verbatim}
Do them like this\marginpar{margin note}
\end{verbatim}
Do them like this\marginpar{margin note}

\item \textbf{Cross References} :-
At the end of this source file is the line
\verb|\label{THE_END}|.
\verb|\pageref{THE_END}| will refer to the final page by number, and 
\verb|\ref{THE_END}| will refer to it by section number. The 
last page (page \pageref{THE_END}) is in section \ref{THE_END}.
You'll have to run \LaTeX\ at least twice to pick up forward references
like these.

\item \textbf{Boxed Text} :-
For short pieces of text, use \verb|\fbox|
\begin{verbatim}
   Help. I'm \fbox{trapped}
\end{verbatim}
   Help. I'm \fbox{trapped}

\item \textbf{Comments} :- Anything to the right of a \% character is ignored by \LaTeX\ .

\item \textbf{Rules} :- The \verb|\rule| takes 2 arguments; a width and a 
height, so\\  \verb|\rule{\textwidth}{1pt}| produces

\noindent\rule{\textwidth}{1pt}
\end{itemize}



\section{Color and Fonts}
\subsection{Colored text}
Commands that control foreground and background
colors  need 
\begin{verbatim}
   \usepackage[dvips]{color}
\end{verbatim}
after the \verb|\documentclass| line but before \verb|\begin{document}|.
\begin{itemize}
\item \verb|\textcolor{|\textit{colorname}\verb|}{|\textit{text}\verb|}|
writes text in a color which can be specified by name (\texttt{black, white,
red, green, blue} or a color name you've defined), RGB components,
or grayscale.
\item \verb|\colorbox{|\textit{colorname}\verb|}{|\textit{text}\verb|}|
writes text in a box with a colored background.
\item \verb|\fcolorbox{|\textit{colorname}\verb|}{|\textit{text}\verb|}|
writes text in a colored frame.
\item \verb|\pagecolor{|\textit{colorname}\verb|}| sets the color
of the page's background.
\item \verb|\definecolor{|\textit{colorname}\verb|}{|\textit{color specification}\verb|}| lets you define new color names.
\end{itemize}

\begin{small}
\begin{verbatim}
\definecolor{gold}{rgb}{0.85,.66,0}
This is in \textcolor{red}{red} and this box is \colorbox{gold}{gold}.
Text color can be set using RGB values 
(\textcolor[rgb]{0,1,0}{like so}), or \textcolor[gray]{0.2}{shades} 
\textcolor[gray]{0.5}{of} \textcolor[gray]{0.8}{grey}.
\end{verbatim}
\end{small}
produces

\definecolor{gold}{rgb}{0.85,.66 , 0}
This is in \textcolor{red}{red} and this box is \colorbox{gold}{gold}.
Text color can be set using RGB values 
(\textcolor[rgb]{0,1,0}{like so}), or \textcolor[gray]{0.2}{shades}
\textcolor[gray]{0.5}{of} \textcolor[gray]{0.8}{grey}.
 

\subsection{Special characters}
   \dag\ is created by \verb|\dag|, \ddag\ by \verb|\ddag|, \S\  by \verb|\S|, \P\ by \verb|\P|, \pounds\ by \verb|\pounds|, \"{o} by \verb|\"{o}|, 
\copyright\ by \verb|\copyright|. 
Many others are available 
in the \texttt{math} environment, including all the lower case greek letters and
most of the upper case ones. If you only want to use a few characters you
can bracket the symbols using \verb|$| and \verb|$| rather than 
\verb|\begin{math}|
and \verb|\end{math}|. You can put a slash through any of these characters by prefacing them with \verb|\not|

\begin{small}
\begin{longtable}{llllll}
$\sqrt{i} $&        \verb|\sqrt{i}| &                   
$\sqrt[5]{x+iy} $&  \verb|\sqrt[5]{x+iy}| &      
$\ldots $&          \verb|\ldots|\\              
$\cdots $&         \verb|\cdots| &             
$\vdots$&           \verb|\vdots|&               
$\ddots$&           \verb|\ddots|\\               
$\alpha$&           \verb|\alpha|&               
$\beta$&           \verb|\beta|&               
$\gamma$&           \verb|\gamma|\\               
$\delta$&           \verb|\delta|&               
$\omega$&           \verb|\omega|&               
$\Gamma$&          \verb|\Gamma|\\              
$\Theta$&           \verb|\Theta|&               
$\Omega$&           \verb|\Omega|&               
$\pm$&              \verb|\pm|\\                  
$\mp$&             \verb|\mp|&                 
$\times$&           \verb|\times|&               
$\div$&             \verb|\div|\\                 
$\ast$&             \verb|\ast|&                 
$\star$&           \verb|\star|&               
$\circ$&            \verb|\circ|\\                
$\bullet$&          \verb|\bullet|&              
$\cdot$&            \verb|\cdot|&                
$\cap$&            \verb|\cap|\\                
$\bigcap$&          \verb|\bigcap|&              
$\cup$&             \verb|\cup|&                 
$\bigcup$&          \verb|\bigcup|\\              
$\uplus$&          \verb|\uplus|&              
$\biguplus$&        \verb|\biguplus|&            
$\sqcap$&           \verb|\sqcap|\\               
$\sqcup$&          \verb|\sqcup|&              
$\bigsqcup$&        \verb|\bigsqcup|&            
$\vee$&             \verb|\vee|\\                 
$\bigvee$&          \verb|\bigvee|&              
$\wedge$&          \verb|\wedge|&              
$\bigwedge$&        \verb|\bigwedge|\\            
$\setminus$&        \verb|\setminus|&            
$\wr$&              \verb|\wr|&                  
$\diamond$&        \verb|\diamond|\\            
$\bigtriangleup$&   \verb|\bigtriangleup|&       
$\bigtriangledown$& \verb|\bigtriangledown|&     
$\triangleleft$&    \verb|\triangleleft|\\        
$\triangleright$&  \verb|\triangleright|&      
%$\lhd$&             \verb|\lhd|&                 
%$\rhd$&             \verb|\rhd|\\                 
%$\unlhd$&           \verb|\unlhd|&               
%$\unrhd$&          \verb|\unrhd|&              
$\oplus$&           \verb|\oplus|\\               
$\bigoplus$&        \verb|\bigoplus|&            
$\ominus$&          \verb|\ominus|&            
$\otimes$&         \verb|\otimes|\\           
$\bigotimes$&       \verb|\bigotimes|&           
$\oslash$&          \verb|\oslash|&              
$\odot$&            \verb|\odot|\\                
$\bigodot$&        \verb|\bigodot|&            
$\bigcirc$&         \verb|\bigcirc|&          
$\amalg$&           \verb|\amalg|\\               
$\leq$&             \verb|\leq|&             
$\prec$&           \verb|\prec|&               
$\preceq$&          \verb|\preceq|\\              
$\ll$&              \verb|\ll|&                  
$\subset$&          \verb|\subset|&              
$\subseteq$&       \verb|\subseteq|\\           
%$\sqsubset$&        \verb|\sqsubset|&            
%$\sqsubseteq$&      \verb|\sqsubseteq|&          
$\in$&              \verb|\in|&                  
$\vdash$&          \verb|\vdash|\\              
$\geq$&             \verb|\geq|&                 
$\succ$&            \verb|\succ|&                
$\succeq$&          \verb|\succeq|\\              
$\gg$&             \verb|\gg|&                 
$\supset$&          \verb|\supset|&              
$\supseteq$&        \verb|\supseteq|\\            
%$\sqsupset$&        \verb|\sqsupset|&            
$\sqsupseteq$&     \verb|\sqsupseteq|&
$\ni$&              \verb|\ni|&                  
$\dashv$&           \verb|\dashv|\\           
$\equiv$&           \verb|\equiv|&               
$\sim$&            \verb|\sim|&                
$\simeq$&           \verb|\simeq|\\               
$\asymp$&           \verb|\asymp|&               
$\approx$&          \verb|\approx|&              
$\cong$&           \verb|\cong|\\            
$\neq$&             \verb|\neq|&                 
$\doteq$&           \verb|\doteq|&               
$\propto$&          \verb|\propto|\\              
$\models$&         \verb|\models|&             
$\perp$&            \verb|\perp|&                
$\mid$&             \verb|\mid|\\               
$\parallel$&       \verb|\parallel|&           
$\bowtie$&         \verb|\bowtie|&             
% $\Join$&            \verb|\Join|\\
$\smile$&           \verb|\smile|\\              
$\frown$&           \verb|\frown|&               
$\leftarrow$&      \verb|\leftarrow|&
$\Leftarrow$&       \verb|\Leftarrow|\\           
$\rightarrow$&      \verb|\rightarrow|&          
$\Rightarrow$&      \verb|\Rightarrow|&          
$\leftrightarrow$& \verb|\leftrightarrow|\\     
$\Leftrightarrow$&  \verb|\Leftrightarrow|&      
$\mapsto$&          \verb|\mapsto|&           
$\hookleftarrow$&   \verb|\hookleftarrow|\\    
$\leftharpoonup$&  \verb|\leftharpoonup|&     
$\leftharpoondown$& \verb|\leftharpoondown|&     
$\rightleftharpoons$& \verb|\rightleftharpoons|\\ 
$\longleftarrow$&   \verb|\longleftarrow|&
$\Longleftarrow$&  \verb|\Longleftarrow|&    
$\longrightarrow$&  \verb|\longrightarrow|\\      
$\Longrightarrow$&  \verb|\Longrightarrow|&      
$\longleftrightarrow$& \verb|\longleftrightarrow|&
$\Longleftrightarrow$& \verb|\Longleftrightarrow|\\
$\longmapsto$&      \verb|\longmapsto|&
$\hookrightarrow$&  \verb|\hookrightarrow|&      
$\rightharpoonup$&  \verb|\rightharpoonup|\\     
$\rightharpoondown$& \verb|\rightharpoondown|& 
%$\leadsto$&         \verb|\leadsto|&             
$\uparrow$&         \verb|\uparrow|&   
$\Uparrow$&         \verb|\Uparrow|\\             
$\downarrow$&      \verb|\downarrow|&          
$\Downarrow$&       \verb|\Downarrow|&           
$\updownarrow$&     \verb|\updownarrow|\\         
$\nearrow$&         \verb|\nearrow|&
$\searrow$&        \verb|\searrow|&         
$\swarrow$&         \verb|\swarrow|\\             
$\nwarrow$&         \verb|\nwarrow|&            
$\aleph$&           \verb|\aleph|&             
$\hbar$&           \verb|\hbar|\\ 
$\imath$&           \verb|\imath|&               
$\jmath$&           \verb|\jmath|&               
$\ell$&             \verb|\ell|\\               
$\wp$&             \verb|\wp|&               
$\Re$&              \verb|\Re|&                  
$\Im$&              \verb|\Im|\\                  
%$\mho$&             \verb|\mho|\\                 
$\prime$&          \verb|\prime|&
$\empty$&           \verb|\empty|&               
$\nabla$&           \verb|\nabla|\\               
$\surd$&            \verb|\surd|&             
$\top$&            \verb|\top|&                
$\bot$&             \verb|\bot|\\                
$\|$&               \verb?\|?&                 
$\angle$&           \verb|\angle|&               
$\forall$&         \verb|\forall|\\             
$\exists$&          \verb|\exists|&              
$\neg$&             \verb|\neg|&                 
$\flat$&            \verb|\flat|\\              
$\natural$&        \verb|\natural|&            
$\sharp$&           \verb|\sharp|&               
$\backslash$&        \verb|\backslash|\\            
$\partial$&         \verb|\partial|&          
$\infty$&          \verb|\infty|&              
%$\Box$&             \verb|\Box|&                 
%$\Diamond$&         \verb|\Diamond|\\             
$\triangle$&        \verb|\triangle|\\
$\clubsuit$&       \verb|\clubsuit|&           
$\diamondsuit$&     \verb|\diamondsuit|&         
$\heartsuit$&       \verb|\heartsuit|\\         
$\spadesuit$&       \verb|\spadesuit|&           
$\sum$&            \verb|\sum|&         
$\prod$&            \verb|\prod|\\                
$\coprod$&          \verb|\coprod|&              
$\int$&             \verb|\int|&                
$\oint$&           \verb|\oint|\\               
\end{longtable}
\end{small}


%One locally produced font with only one character is called \texttt{crest}.
%\begin{verbatim}
%\font\crest=crest
%``This is the {\crest A} CUED crest''
%\end{verbatim}
%produces 
%\font\crest=crest
%``This is the {\crest A} CUED crest''
\subsection{Font Sizes}
These are the available sizes.

\begin{tabular}{lllll}
{\tiny tiny} & {\scriptsize scriptsize} & {\footnotesize footnotesize} & 
{\small small} & {\normalsize normalsize} \\  {\large large} & {\Large Large} &
{\LARGE LARGE} & {\huge huge} & {\Huge Huge} \\
\end{tabular}

If, for example, you want to use the smallest size, do
\begin{verbatim}
{\tiny ... }
\end{verbatim}

If \texttt{Huge} isn't big enough for you, you can scale a \emph{postscript}
font up using the commands designed for graphics. 
\verb|\resizebox{!}{5cm}{BIG}| produces

\resizebox{!}{5cm}{BIG}


\subsection{Font Types}
Independent of size, these font types are at your disposal :-
\verb|\textrm| \textrm{(roman)}, \verb|textit|\textit{(italic)}, 
\verb|\textsc| \textsc{(small caps)}, \verb|\emph| \emph{(emphasis, but note that
if you use \texttt{emph} within emphasized text, you will get \emph{roman text})},
\verb|\textsl| \textsl{(slanting)}, \verb|\texttt| \texttt{(teletype)}, 
\verb|\textbf| \textbf{(boldface)},
\verb|\textsf| \textsf{(sans serif)}. As long as there's no conflict, these
commands can be combined so that, for instance,  \textsf{\textbf{this is bold sans serif}} can be produced by 
\verb|\textsf{\textbf{this is bold sans serif}}|.

\subsection{Postscript Fonts}
It is easy to write a document that has postscript fonts. We have
package support for helvetica (\texttt{helvetic}), utopia, times, optima, 
newcentury (\texttt{newcentu}), palatino and courier. To use 
palatino, for instance, all you need to do is add
\begin{verbatim}
\usepackage{palatino}
\end{verbatim}
to your file. The \texttt{pifont} package has special commands for using the Zapf Dingbats
font. \verb|\dingfill{40}| completes the line with the specified
symbol \dingfill{40} 

and \verb|\dingline{36}| draws a whole line of
symbols. \dingline{36}


It's a good idea to use a font that's installed in the
printer you intend to use. See the table on page \pageref{LJMR1LIST}
for information about the teaching system's \texttt{ljmr1} printer.

\subsection{Font attributes}
\emph{The commands above should give you sufficient control over fonts. If
you don't want to know more at the moment then turn to section 
\ref{ENVIRONMENTS}}

Every text font in \LaTeX{} has five \emph{attributes}:
\begin{description}

\item[encoding] This specifies the order that characters appear in the
   font. The most common values for the font encoding is \texttt{OT1}.
  
\item[family] The name for a collection of fonts, usually grouped under
   a common name by the font foundry.  For example, `Adobe Times' and 
Knuth's `Computer Modern Roman' are font families.
There are far too many font families to list them all, but some common
ones are:


\begin{center}
\label{LJMR1LIST}
   \begin{tabular}{>{\ttfamily}lll}
      \emph{Internal fontname} & \emph{Fontname} & \emph{In ljmr1?}\\
      cmr  & Computer Modern Roman & No\\
      cmss & Computer Modern Sans & No\\
      cmtt & Computer Modern Typewriter & No\\
      cmm  & Computer Modern Math Italic & No\\
      cmsy & Computer Modern Math Symbols & No\\
      cmex & Computer Modern Math Extensions & No\\
      ptm  & Adobe Times & Yes\\
      phv  & Adobe Helvetica & Yes \\
      pcr  & Adobe Courier& Yes \\
      pun  & Univers& No\\
      ppl  & Palatino& Yes \\
      pagk  & AvantGarde-Book& Yes\\
      pagd  & AvantGarde-Demi& Yes\\ 
      pbk  & Bookman& Yes\\
      put & Utopia& No\\
      pop & Optima& No\\
      pnc & New Century Schoolbook& Yes\\
      pzd & ZapfDingbats & Yes\\
      rpad & Garamond & No\\
   \end{tabular}  
\end{center}


\item[series] How heavy or expanded a font is.  For example, `medium
   weight', `narrow' and `bold extended' are all series.
The most common values for the font series are:
\begin{center}
\begin{minipage}{.7\linewidth}
   \begin{tabular}{>{\ttfamily}rl}
      m    & Medium  \\
      b    & Bold  \\
      bx   & Bold extended \\
      sb   & Semi-bold \\
      c    & Condensed
   \end{tabular}
\end{minipage}
\end{center}

\item[shape] The form of the letters within a font family.  For
    example,    `italic', `oblique' and `upright' are all font shapes.
The most common values for the font shape are:
\begin{center}
\begin{minipage}{.7\linewidth}
   \begin{tabular}{>{\ttfamily}rl}
      n    & Normal (that is `upright' or `roman') \\
      it   & Italic \\
      sl   & Slanted (or `oblique') \\
      sc   & Caps and small caps
   \end{tabular}
\end{minipage}
\end{center}


\item[size] The design size of the font, for example `10pt'.

\end{description}

These five parameters specify every \LaTeX{} font,
for example:
\begin{small}
\begin{center}
   \begin{tabular}{llllll}
      \multicolumn{5}{l}\emph{\LaTeX{} specification} &
      \emph{Font} \\
      OT1 & cmr & m & n & 10pt &
      Computer Modern Roman 10pt \\
      OT1 & cmss & m & sl & 12pt &
      Computer Modern Sans Oblique 12pt \\
      OML & cmm & m & it & 10pt &
      Computer Modern Math Italic 10pt \\
      T1 & ptm & b & it & 18pt &
      Adobe Times Bold Italic 18pt\\
   \end{tabular}
\end{center}
\end{small}

\subsection{Selection commands}
There are commands to set attributes one at a time:
\begin{center}
   \begin{tabular}{lll}
      \emph{Command} &
      \emph{Attribute} &
      \emph{Value in} article \emph{class}, 10pt \\
      \verb|\textrm{..}| or \verb|\rmfamily|    & family & \verb|cmr| \\
      \verb|\textsf{..}| or \verb|\sffamily|    & family & \verb|cmss| \\
      \verb|\texttt{..}| or \verb|\ttfamily|    & family & \verb|cmtt| \\
      \verb|\textmd{..}| or \verb|\mdseries|    & series & \verb|m| \\
      \verb|\textbf{..}| or \verb|\bfseries|    & series & \verb|bx| \\
      \verb|\textup{..}| or \verb|\upshape|     & shape  & \verb|n| \\
      \verb|\textit{..}| or \verb|\itshape|     & shape  & \verb|it| \\
      \verb|\textsl{..}| or \verb|\slshape|     & shape  & \verb|sl| \\
      \verb|\textsc{..}| or \verb|\scshape|     & shape  & \verb|sc| \\
      \verb|\tiny|         & size   & \verb|5pt| \\
      \verb|\scriptsize|   & size   & \verb|7pt| \\
      \verb|\footnotesize| & size   & \verb|8pt| \\
      \verb|\small|        & size   & \verb|9pt| \\
      \verb|\normalsize|   & size   & \verb|10pt| \\
      \verb|\large|        & size   & \verb|12pt| \\
      \verb|\Large|        & size   & \verb|14.4pt| \\
      \verb|\LARGE|        & size   & \verb|17.28pt| \\
      \verb|\huge|         & size   & \verb|20.74pt| \\
      \verb|\Huge|         & size   & \verb|24.88pt|
   \end{tabular}
\end{center}



The low-level commands used to change font attributes are as follows.

\begin{tabular}{l}
\verb|\fontencoding{|\textit{encoding}\verb|}| \\
\verb|\fontfamily{|\textit{family}\verb|}|\\
\verb|\fontseries{|\textit{series}\verb|}|\\
\verb|\fontshape{|\textit{shape}\verb|}|\\
\verb|\fontsize{|\textit{size}\verb|}{|\textit{baselineskip}\verb|}|
\end{tabular}

Each of these commands sets one of the font
attributes; \verb|\fontsize| also sets \verb|\baselineskip|. The
actual font in use is not altered by these commands, but the current
attributes are used to determine which font to use after the next
\verb|\selectfont| command.

\verb|\selectfont|
selects a text font, based on the current values of the font attributes.
There \emph{must} be a \verb|\selectfont| command
immediately after any settings of the font parameters by (some of)
the five \verb|\font<parameter>| commands, before any following text.
For example, it is legal to say:
\begin{verbatim}
   \fontfamily{ptm}\fontseries{b}\selectfont Some text.
\end{verbatim}
to select bold Times Roman, but it is \emph{not} legal to say:
\begin{verbatim}
   \fontfamily{ptm} Some \fontseries{b}\selectfont text.
\end{verbatim}

\verb|\usefont{|\emph{encoding}\verb|}{|\emph{family}\verb|}{|\emph{series}\verb|}{|\emph{shape}\verb|}|

\noindent is short hand for the equivalent \verb|\font|\ldots{} commands followed by\verb|\selectfont|.

\section{Environments}
\label{ENVIRONMENTS}
Some examples of how to use environments are given here.

\subsection{Alignments}
In these environments \verb|\\| starts a new line.

\begin{verbatim}
\begin{flushleft}
Some people like to stay firmly\\ on the left whereas others
\end{flushleft}
\begin{flushright}
feel more at home on the right.\\ 
\end{flushright}
\begin{center}
but most of us prefer to stay dead in the center.
\end{center}
\end{verbatim}

\begin{flushleft}
Some people like to stay firmly\\ on the left whereas others
\end{flushleft}
\begin{flushright}
feel much more at home\\ on the right.\\ 
\end{flushright}
\begin{center}
but most of us prefer to stay dead in the center.
\end{center}

\subsection{Listing Items}
The items can be marked in one of three way:
\begin{verbatim}
\begin{itemize}
\item just by a bullet, using \texttt{itemize}

\item numbered, using \texttt{enumerate} 

\begin{enumerate}
\item one
\item two
\item three
\end{enumerate}

\item or with a label, using \texttt{description} 

\begin{description}
\item[itemize] bullets
\item[enumerate] automatic numbering
\item[description] labelling
\end{description}

\end{itemize}

\end{verbatim}
\begin{itemize}
\item just by a bullet, using \texttt{itemize}

\item numbered, using \texttt{enumerate} 

\begin{enumerate}
\item one
\item two
\item three
\end{enumerate}

\item or with a label, using \texttt{description}  

\begin{description}
\item[itemize] bullets
\item[enumerate] automatic numbering
\item[description] labelling
\end{description}

\end{itemize}

The \texttt{pifont} package includes a variant of the \texttt{itemize} 
command that will 
replace the usual 'bullet' by a Zapf Dingbat symbol of your choice
\begin{verbatim}
\begin{dinglist}{43}
\item First 
\item Second
\end{dinglist}
\end{verbatim}

\begin{dinglist}{43}
\item First 
\item Second
\end{dinglist}

and a variant of the \texttt{enumerate} command that given an 
initial Zapf Dingbat symbol will increment the symbol for each item.
\begin{verbatim}
\begin{dingautolist}{172}
\item First 
\item Second
\end{dingautolist}
\end{verbatim}
\begin{dingautolist}{172}
\item First 
\item Second
\end{dingautolist}

\subsection{Tabular}
Tabular output is supported. When you create the environment you specify how 
many columns to have and how the contents are to be aligned (use \texttt{l, c} 
or \texttt{r} to represent each column with either left, center or right 
alignment) and where you want vertical lines (use \verb?|?). The contents of 
the columns are separated by a `\&' and rows by \verb|\\|. Here's a simple
example
\begin{verbatim}
\begin{tabular}{l|c|r}
left & centre & right\\
more left & more centre & more right\\
\end{tabular}
\end{verbatim}

\begin{tabular}{l|c|r}
left & centre & right\\
more left & more centre & more right\\
\end{tabular}

To draw a full 
horizontal line, use \verb|\hline| otherwise draw a line across selected 
columns using \verb|\cline|. The \verb|\multicolumn| command allows items to 
span columns. It takes as its first argument the number of columns to span.
The following, more complicated example shows how to use these 
facilities.
\begin{verbatim}
\begin{tabular}{||l|lr||} \hline
\textbf{Veg}  & \multicolumn{2}{|c||}{\textbf{Detail}}\\\hline
carrots    & per pound & \pounds 0.75 \\ \cline{2-3}
           & each      &         20p  \\ \hline
mushrooms  & dozen     &         86p  \\ \cline{1-1} \cline{3-3}
toadstools & pick your own & free     \\ \hline
\end{tabular}
\end{verbatim}

\begin{tabular}{||l|lr||} \hline
\textbf{Veg}  & \multicolumn{2}{|c||}{\textbf{Detail}}\\\hline
carrots    & per pound & \pounds 0.75 \\ \cline{2-3}
           & each      &         20p  \\ \hline
mushrooms  & dozen     &         86p  \\ \cline{1-1} \cline{3-3}
toadstools & pick your own & free     \\ \hline
\end{tabular}

Tables won't continue on the next page if they're too long. The 
\texttt{longtable} or \texttt{supertabular} commands  are needed to do this. 
See the \htmladdnormallinkfoot{Supertabular}{http://www-h.eng.cam.ac.uk/help/tpl/textprocessing/supertab.dvi} document for details and examples.

If the text in a column is too wide for the page, \LaTeX\ won't 
automatically text-wrap. Using \verb|p{5cm}| instead of  \verb|c|,
 \verb|l| or  \verb|r| in the \texttt{tabular} line will wrap-around
the text in a 5 cm wide column.

There are various packages to assist with table creation. The 
\texttt{array} package adds some helpful features, including the ability
to add formatting commands that control a whole column at a time, like
so
\begin{verbatim}
\begin{tabular}{>{\ttfamily}l>{\scshape}c>{\Large}r}
      Text  & More Text & Large Text\\
      Left  & Centred   & Right
\end{tabular}
\end{verbatim}

\begin{tabular}{>{\ttfamily}l>{\scshape}c>{\Large}r}
      Text  & More Text & Large Text\\
      Left  & Centred   & Right
\end{tabular}

The 
\htmladdnormallinkfoot{\texttt{rotating}}{http://www-h.eng.cam.ac.uk/help/tpl/textprocessing/rotating.ps}
package is useful if you have a wide table that you
want to display in landscape mode. You need to put your table inside
\verb|\begin{sidewaystable}| and \verb|\end{sidewaystable}|.

If you want the table to have a caption and \emph{float} (float up the
page if it's started right near the foot of a page, for example), use
\begin{verbatim}
\begin{table}[htbp] 
\begin{tabular}...
...
\end{tabular}
\caption{...}
\end{table}
\end{verbatim}
See section \ref{FIGURES} for details.

\subsection{Array}
The \texttt{array} environment (not to be confused with the \texttt{array}
package) is similar to the \texttt{tabular} but must be within a 
\emph{math} environment. This
\begin{verbatim}
\begin{math}
\left(
\begin{array}{clrr}
      a+b+c & uv & x-y & 27 \\
       x+y  & w  & +z  & 363 
\end{array}
\right)
\end{math}
\end{verbatim}
produces
\begin{math}
\left(
\begin{array}{clrr}
      a+b+c & uv & x-y & 27 \\
       x+y  & w  & +z  & 363 
\end{array}
\right)
\end{math}

\subsection{Pictures}
\LaTeX\ has some graphics capabilities. It's much better to import an
encapsulated postscript file.
See the \htmladdnormallinkfoot{\textit{\LaTeX\ Maths and Graphics}}
{http://www-h.eng.cam.ac.uk/help/tpl/textprocessing/latex\_maths+pix/latex\_maths+pix.html} document for more details.
\begin{verbatim}
\newcounter{cms}
\setlength{\unitlength}{1mm}
\begin{picture}(50,39)
\put(0,7){\makebox(0,0)[bl]{cm}}
\multiput(10,7)(10,0){5}{\addtocounter
   {cms}{1}\makebox(0,0)[b]{\arabic{cms}}}
\put(15,20){\circle{6}}
\put(30,20){\circle{6}}
\put(15,20){\circle*{2}}
\put(30,20){\circle*{2}}
\put(10,24){\framebox(25,8){a box}}
\put(10,32){\vector(-2,1){10}}
\multiput(1,0)(1,0){49}{\line(0,1){2,5}}
\multiput(5,0)(10,0){5}{\line(0,1){3,5}}
\thicklines
\put(0,0){\line(1,0){50}}
\multiput(0,0)(10,0){6}{\line(0,1){5}}
\end{picture}
\end{verbatim}

\newcounter{cms}
\setlength{\unitlength}{1mm}
\begin{picture}(50,39)
\put(0,7){\makebox(0,0)[bl]{cm}}
\multiput(10,7)(10,0){5}{\addtocounter
   {cms}{1}\makebox(0,0)[b]{\arabic{cms}}}
\put(15,20){\circle{6}}
\put(30,20){\circle{6}}
\put(15,20){\circle*{2}}
\put(30,20){\circle*{2}}
\put(10,24){\framebox(25,8){a box}}
\put(10,32){\vector(-2,1){10}}
\multiput(1,0)(1,0){49}{\line(0,1){2,5}}
\multiput(5,0)(10,0){5}{\line(0,1){3,5}}
\thicklines
\put(0,0){\line(1,0){50}}
\multiput(0,0)(10,0){6}{\line(0,1){5}}
\end{picture}

\subsection{Maths}
Maths is dealt with in the \htmladdnormallinkfoot{\textit{\LaTeX\ Maths and Graphics}}
{http://www-h.eng.cam.ac.uk/help/tpl/textprocessing/latex\_maths+pix/latex\_maths+pix.html} document. Here are some examples
\begin{verbatim}
\begin{math}
 \lim_{n \rightarrow \infty}x = 0 \\
x^{2y} \\
x_{2y} \\
x^{2y}_{1} \\
\frac{x+y}{1 + \frac{1}{n+1}} \\
\end{math}
\end{verbatim}
produces
\begin{math}
 \lim_{n \rightarrow \infty}x = 0 \\
x^{2y} \\
x_{2y} \\
x^{2y}_{1} \\
\frac{x+y}{1 + \frac{1}{n+1}}
\end{math}

\subsection{Figures}
\label{FIGURES}
To include graphics, use the \verb|\includegraphics| command
(provided by the \texttt{graphicx} package) inside \texttt{figure} environment. 
The arguments to
 \texttt{figure} specify where the space will be made, preferentially

\begin{tabular}{>{\ttfamily}ll}
h & here \\
t & top of page \\
b & bottom of page \\
p & on a page with no text \\
\end{tabular}

\begin{verbatim}
\begin{figure}[htbp]
   \vspace{1cm}
   \caption{1 cm of space}
\end{figure}
\end{verbatim}
\begin{figure}[htbp]
   \vspace{1cm}
   \caption{1 cm of space}
\end{figure}
Putting \texttt{!} as the first argument in the square brackets will 
encourage  \LaTeX\ to
do what you say, even if the result's sub-optimal.

If you have a label defined in the caption, \LaTeX\ may
give an error message. \verb|\label| is a \emph{fragile} command (see the 
\LaTeX\ book for details) so you'll need to do something like
\begin{verbatim}
...
\caption{1 cm of space\protect\label{EMPTY}}
\end{verbatim}
or simply put the \verb|\label| command after the caption. Note that
if you put the \verb|\label| \emph{before} the caption, the resulting reference will be the
section number and not the figure number. See the 
\htmladdnormallinkfoot{\textit{\LaTeX\ Maths and Graphics}}
{http://www-h.eng.cam.ac.uk/help/tpl/textprocessing/latex\_maths+pix/latex\_maths+pix.html} document for more details.

\subsection{Tabbing}
Within this environment tabs can be set by \verb|\=| and the next tab
moved to by using \verb|\>|.
\begin{verbatim}
\begin{tabbing}
if \= it's raining\\      % set tab here, after the 'if'
   \> get an umbrella \=\\ % go to the defined tab and set a new one
else\\
   \> get wet \> * next tab is here\\
endif
\end{tabbing}
\end{verbatim}
\begin{tabbing}
if \= it's raining\\      % set tab here, after the 'if'
   \> get an umbrella \=\\ % go to the defined tab and set a new one
else\\
   \> get wet \> * next tab is here\\
endif
\end{tabbing}

\subsection{Verbatim}
Within this environment things come out unformatted.
It's useful for showing examples of typed input and provides
a way of printing characters that have a special meaning for 
\LaTeX.

\verb?\begin{verbatim}?
\verb?caret is ^, tilde is ~ and backslash is \?
\verb?\end{verbatim}?

produces

\begin{verbatim}
caret is ^, tilde is ~ and backslash is \ 
\end{verbatim}
If you just want to quote a few characters, use \verb?\verb|?\textit{quoted text}\verb?|?.
The characters delimiting the quote can be anything as long as they are the same.

\subsection{Quote, abstract}
These widen the margins and change the font. The \texttt{abstract} environment
also adds a title.
\subsection{Letter}
See \texttt{letter.tex} in \texttt{/export/Examples/LaTeX} 

\subsection{Curriculum Vitae}
See \texttt{cv.tex} in \texttt{/export/Examples/LaTeX} 
 
\section{Customising}
\subsection{Macros}
At the top of the source of this file is
\begin{verbatim}
\def\xdt{$\cal X\!\!$\texttt{.desktop}}
\end{verbatim}
which defines \verb|\xdt| to be \xdt . Using such constructions can make
your document much tidier, and saves on typing.

\subsection{Modifications}
Many of the features of a \LaTeX{} document are easily customised, but
you'll often have to look at the class (\verb|.cls|) files to find out what to
do. For example, suppose you wanted to have \textbf{References} in a book
rather than \textbf{Bibliography}. If you look in \texttt{book.cls} you'll
see
\begin{verbatim}
   \newcommand\bibname{Bibliography}
\end{verbatim}
so adding
\begin{verbatim}
   \renewcommand{\bibname}{References}
\end{verbatim}
to your file should achieve what you want.

Counters (\emph{e.g.} \texttt{figure}) have related commands 
(\emph{e.g.} \texttt{thefigure}) to control their appearance, so 
they're easy to customise
\begin{verbatim} 
   \renewcommand\thefigure{\roman{figure}}
\end{verbatim}
produces figure numbers in lower case roman numerals. 
Longer commands can be adapted too. Remember however, that if the
command involves a \verb|@| you have to enclose your changes in
\verb|\makeatletter| ... \verb|\makeatother|. Here's an example that
changes the appearance of section headings - 
\begin{verbatim} 
\makeatletter
\renewcommand{\section}{\@startsection{section}{1}{0mm}
{\baselineskip}%
{\baselineskip}{\normalfont\normalsize\scshape\centering}}%
\makeatother
\begin{document}
\end{verbatim}



\subsection{New Commands}

A completely new command can be created using

\verb|\newcommand{\|
\textit{commandname}\verb|[|\textit{number of arguments}\verb|]{|
\textit{command text, using} \texttt{\#1}, \texttt{\#2} \textit{etc to denote arguments}\verb|}|

For example,
\begin{verbatim}
\newcommand{\ve}[1]{\(#1_1 ... #1_n\)}
\ve{x}
\end{verbatim}
produces as output
\newcommand{\ve}[1]{\(#1_1 ... #1_n\)}
\ve{x}

A problem with this example is that it shouldn't change to math mode if 
\LaTeX{} is already in that mode. A better try would be
\begin{verbatim}
\newcommand{\ve}[1]{\ensuremath{#1_1 ... #1_n}}
\end{verbatim}
which will only change to math mode if it's necessary.

A new environment is just as easily created - give the name of the
environment, and what you want to happen on entering and leaving
the environment. The following provides a
variant of the itemize command.
\begin{verbatim}
\newenvironment{emlist}{\begin{itemize} \em}{\end{itemize}}
\begin{emlist}
\item first comment
\item second comment
\end{emlist}
The end of the environment ends the scope of the emphasis.
\end{verbatim}
\newenvironment{emlist}{\begin{itemize} \em}{\end{itemize}}
\begin{emlist}
\item first comment
\item second comment
\end{emlist}
The end of the environment ends the scope of the emphasis.

\subsection{Packages}
There are many features and options not mentioned in this handout. See
the 
\htmladdnormallinkfoot{packages}
{http://www-h.eng.cam.ac.uk/help/tpl/textprocessing/LaTeX\_intro.html\#Packages} section of the online \LaTeX{} page for more details. You can 
often find out what you want by looking at the files in the system directories.
 If your file 
begins 
\begin{verbatim}
\documentclass[12pt]{article}
\end{verbatim}
then various macro files are read 
when you run \texttt{latex}, namely (\texttt{article.cls} and  \texttt{size12.sty}). Some of these files are 
well-enough commented to be useful documentation. 

\subsection{An Example}

\texttt{minipage} creates a miniature page complete with its own footnotes
etc. It can be created to any set width. Footnotes within a 
minipage use a different counter to other footnotes\footnote{like
this one}. This example ensures that all the numbers follow on in sequence. 
The footnotes in minipages are marked by letters rather than numbers, so 
here the type is changed to arabic.
\begin{verbatim}
\begin{minipage}{\textwidth}
% Set the minipage footnote counter
\setcounter{mpfootnote}{\value{footnote}}
% Redefine the command that produces the footnote number
\renewcommand{\thempfootnote}{\arabic{mpfootnote}}
\begin{tabular}{|ll|}\hline
one & two\footnote{A minipage footnote}\\\hline
\end{tabular}
\setcounter{footnote}{\value{mpfootnote}}
\end{minipage}
\end{verbatim}
\begin{minipage}{\textwidth}
% Set the minipage footnote counter
\setcounter{mpfootnote}{\value{footnote}}
% Redefine the command that produces the footnote number
\renewcommand{\thempfootnote}{\arabic{mpfootnote}}
\begin{tabular}{|ll|}\hline
one & two\footnote{A minipage footnote}\\\hline
\end{tabular}
\setcounter{footnote}{\value{mpfootnote}}
\end{minipage}

You might find this trick useful if you want footnotes in tables. They
don't come out otherwise.\footnote{Not for me anyway}

Here's an example of using a package. The \texttt{multicol} file
lets one change the number of columns easily. To switch into 3-column
text use
\begin{verbatim}
\begin{multicols}{3}{
Put the text here. Maths, tables, pictures etc are all ok, but not
figures. But you have to remember to load in the \texttt{multicol}
package at the top of your document.
}
\end{multicols}
\end{verbatim}
\begin{multicols}{3}{
Put the text here. Maths, tables, pictures etc are all ok, but not
figures. But you have to remember to load in the \texttt{multicol}
package at the top of your document.
}
\end{multicols}
 \LaTeX\ uses the
value of the environmental variable \texttt{TEXINPUTS} to decide where to
look for \texttt{sty} files. If you create a directory called (say) 
\texttt{inputs}, copy a system \verb|*.sty| file into it and do
\begin{verbatim} 
export TEXINPUTS=inputs:/opt/latex/inputs:.
\end{verbatim}
then your copy will be read in preference to the system one and
you can customise easily.


\section{More Information}
\begin{itemize}
\item See the 
\htmladdnormallinkfoot{LaTeX help page}{http://www-h.eng.cam.ac.uk/help/tpl/textprocessing/LaTeX\_intro.html}.  Much of the material available from there
isn't available in printed form. The guides to producing
\htmladdnormallink{Posters and booklets}{http://www-h.eng.cam.ac.uk/help/tpl/textprocessing/posters.html}
\htmladdnormallink{Reports}{http://www-h.eng.cam.ac.uk/help/tpl/textprocessing/reports.html}
and
\htmladdnormallink{OHP slides}{http://www-h.eng.cam.ac.uk/help/tpl/textprocessing/slides.html}
are useful.
\item See the \htmladdnormallink{comp.text.tex}{news:comp.text.tex} newsgroup.
\item Use the CTAN Archive 
\htmladdnormallink{ftp.tex.ac.uk}{ftp://ftp.tex.ac.uk}
\item Engineering Department users can see the examples in 
\texttt{/export/Examples/LaTeX}. The
\texttt{Tutorial} subdirectory contains exercises. 
\item Engineering Department users can borrow a \LaTeX\ manual from the 
machine room.
\item Look at the \LaTeX\ files in your system. 
\end{itemize}
\label{THE_END}
\end{document}
