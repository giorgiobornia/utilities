% Copyright (c) 2004 by T.P. Love. This document may be copied freely 
% for the purposes of education and non-commercial research.
% Cambridge University Engineering Department,  
% Cambridge CB2 1PZ, England.
% Use latex file.tex ;latex file.tex ; dvips  file.dvi
% to print a copy. 
\documentclass[dvips]{article}
\usepackage{palatino,a4,graphicx,color,subfig,longtable,array}
\usepackage{floatflt,contentsfancybox,html,bm}

\newcommand{\AmS}{${\protect\the\textfont2 A}\kern-.1667em\lower
         .5ex\hbox{\protect\the\textfont2 M}\kern
         -.125em{\protect\the\textfont2 S}$}
 
\newcommand{\AmSLaTeX}{\mbox{\AmS-\LaTeX}}


\def\bydefn{\stackrel{def}{=}}
\def\convf{\hbox{\space \raise-2mm\hbox{$\textstyle  \bigotimes \atop \scriptstyle \omega$} \space}}

\begin{document}

\title{\LaTeX\ maths and graphics}
\author{Tim Love}
\date{\today}
\maketitle

\begin{quotation}
This handout assumes that you have already read the 
\htmladdnormallinkfoot{Advanced LaTeX}{http://www-h.eng.cam.ac.uk/help/tpl/textprocessing/latex\_advanced/latex\_advanced.html}
document handout, so if you're unsure about `environments', read no further.

Note that there's an alternative way of producing maths in \LaTeX{} - 
 \AmSLaTeX{}. See the 
\htmladdnormallinkfoot{online manual}{http://www-h.eng.cam.ac.uk/help/tpl/textprocessing/amslatex.dvi}
 for details.

If you want to more more about graphics, see \htmladdnormallinkfoot{Using Imported Graphics in LaTeX2e Documents}{http://www-h.eng.cam.ac.uk/help/tpl/textprocessing/epslatex.ps}  by Keith Reckdahl.  

Comments and bug reports to Tim Love (\htmladdnormallink{tpl@eng.cam.ac.uk}{mailto:tpl@eng.cam.ac.uk}). 
\end{quotation}

\begin{table}[b]
Copyright \copyright 2004 by T.P.~Love.
This document may be copied freely for the purposes 
of education and non-commercial research.
Cambridge University Engineering Department,  
Cambridge CB2 1PZ, England.
\end{table}


\tableofcontents



\section{Maths}
%\verb| {}| are used to group.
There's more to maths typesetting than meets the eye. Many conventions
used in the typesetting of plain text are inappropriate to maths. \LaTeX\ 
goes a long way to help you along with the style. For example,
in a \LaTeX\ maths environment, letters come out in {\it italics}, \texttt{`-'}
as `$-$' (minus) instead of the usual `-' (dash), `*' becomes $*$,
\verb|'| becomes $'$ and spacing is 
changed (less around `/', more around `+').

Many of the usual \LaTeX\ constructions can still be used in maths environments
but their effect may be slightly different; eg \verb|\textbf{  }| only affects letters 
and numbers. `\verb|{|' and  `\verb|}|' are still special characters;
they're used to group characters.

As usual in \LaTeX\ you can override the defaults, but think before doing
 it: maths support in \LaTeX\ has been carefully thought out and is 
quite logical though the \LaTeX\ source text may not be very readable.
It's a good idea to write out the formulae on paper before you start
\LaTeX ing, and try not to overdo the use of the `\verb|\frac|' 
construction; use `\verb|/|' instead. 



\subsection{Environments}
There are 2 environments to display one-line equations.


\begin{description}

\item[equation:-] Equations in this environment are numbered.
\begin{verbatim}
\begin{equation}
 x + iy
\end{equation}
\end{verbatim}
\begin{equation}
 x + iy
\end{equation}
\item[displaymath:-] These won't be numbered.  \verb|\[, \]| 
can be used as abbreviations for \verb|\begin{displaymath}| and
\verb|\end{displaymath}|.
\begin{verbatim}
\begin{displaymath}
 x + iy
\end{displaymath}
\end{verbatim}
\begin{displaymath}
 x + iy
\end{displaymath}

\end{description}
Never leave a blank line before these equations; it 
starts a new paragraph and looks ugly. 
'\verb|\displaystyle|' is the font type used to print maths in these 
\emph{display} environments. Other relevant environments are:-
\begin{description}
\item[math:-] For use in text. \verb|\(| and  \verb|\)| can be used
to delimit the environment, as can the \TeX\ constructions \verb|$| 
and \verb|$| . For example, \verb|$x=y^2$| gives $x=y^2$.
\item[eqnarray:-]
This is like a 3 column tabular environment. Each line by default is numbered.
You can use  the  \verb|eqnarray*| variant to suppress numbering altogether.
\begin{verbatim}
\begin{eqnarray}
a1 & = & b1 + c1\nonumber\\
a2 & = & b2 - c2
\end{eqnarray}
\end{verbatim}
\begin{eqnarray}
a1 & = & b1 + c1\nonumber\\
a2 & = & b2 - c2
\end{eqnarray}
\end{description}

Maths in these 2 sorts of environments have different default
sizes for some characters and other behavioural differences so 
that a line of maths won't impinge on text lines below or above. 
If you want to put some non-maths text in amongst maths then
enclose it in an \verb|\mbox{...}|.

\subsection{Special Characters}
The \texttt{amssymb} package offers more symbols if the following
aren't enough. The \htmladdnormallinkfoot{symbols}{http://www-h.eng.cam.ac.uk/help/tpl/textprocessing/symbols.ps} document (a postscript file) has a bigger list.

\subsubsection{Greek}
\begin{longtable}{llllllllll}
$\alpha$&           \verb|\alpha|&               
$\beta$&           \verb|\beta|&               
$\gamma$&           \verb|\gamma|&               
$\delta$&           \verb|\delta|&               
$\epsilon$&      \verb|\epsilon|\\ 
$\zeta$&     	 \verb|\zeta|&  
$\eta$&      	 \verb|\eta|&      
$\theta$&    	 \verb|\theta|&    
$\iota$&     	 \verb|\iota|&     
$\kappa$&    	 \verb|\kappa|\\    
$\lambda$&    	 \verb|\lambda|&    
$\mu$&       	 \verb|\mu|&    
$\nu$&       	 \verb|\nu|&       
$\xi$&       	 \verb|\xi|&       
$o$&         	 \verb|o|\\         
$\pi$&       	 \verb|\pi|&       
$\rho$&      	 \verb|\rho|&      
$\sigma$&    	 \verb|\sigma|&    
$\tau$&      	 \verb|\tau|&      
$\upsilon$&  	 \verb|\upsilon|\\  
$\phi$&      	 \verb|\phi|&  
$\chi$&      	 \verb|\chi|&      
$\psi$&       	 \verb|\psi|&      
$\omega$&           \verb|\omega|&
$\Gamma$&           \verb|\Gamma|\\
$\Delta$&           \verb|\Delta|&
$\Theta$&           \verb|\Theta|&
$\Lambda$&          \verb|\Lambda|& 
$\Xi$&              \verb|\Xi| &
$\Pi$&              \verb|\Pi|\\
$\Sigma$&           \verb|\Sigma|&
$\Upsilon$&         \verb|\Upsilon|&
$\Phi$&             \verb|\Phi|&
$\Psi$&             \verb|\Psi| &      
$\Omega$&           \verb|\Omega|\\               
\end{longtable}



\subsubsection{Miscellaneous}

\begin{longtable}{llllll}
$\ldots $&          \verb|\ldots|&              
$\cdots $&         \verb|\cdots| &             
$\vdots$&           \verb|\vdots|\\               
$\ddots$&           \verb|\ddots|&               
$\pm$&              \verb|\pm|&                  
$\mp$&             \verb|\mp|\\                
$\times$&           \verb|\times|&               
$\div$&             \verb|\div|&                 
$\ast$&             \verb|\ast|\\                 
$\star$&           \verb|\star|&               
$\circ$&            \verb|\circ|&                
$\bullet$&          \verb|\bullet|\\              
$\cdot$&            \verb|\cdot|&                
$\cap$&            \verb|\cap|&                
$\bigcap$&          \verb|\bigcap|\\             
$\cup$&             \verb|\cup|&                 
$\bigcup$&          \verb|\bigcup|&              
$\uplus$&          \verb|\uplus|\\           
$\biguplus$&        \verb|\biguplus|&            
$\sqcap$&           \verb|\sqcap|&               
$\sqcup$&          \verb|\sqcup|\\             
$\bigsqcup$&        \verb|\bigsqcup|&            
$\vee$&             \verb|\vee|&                
$\bigvee$&          \verb|\bigvee|\\              
$\wedge$&          \verb|\wedge|&              
$\bigwedge$&        \verb|\bigwedge|&            
$\setminus$&        \verb|\setminus|\\            
$\wr$&              \verb|\wr|&                  
$\diamond$&        \verb|\diamond|&          
$\bigtriangleup$&   \verb|\bigtriangleup|\\       
$\bigtriangledown$& \verb|\bigtriangledown|&     
$\triangleleft$&    \verb|\triangleleft|&       
$\triangleright$&  \verb|\triangleright|\\      
%$\lhd$&             \verb|\lhd|&       
%$\rhd$&             \verb|\rhd|&                 
%$\unlhd$&           \verb|\unlhd|\\               
%$\unrhd$&          \verb|\unrhd|&              
$\oplus$&           \verb|\oplus|&               
$\bigoplus$&        \verb|\bigoplus|&           
$\ominus$&          \verb|\ominus|\\            
$\otimes$&         \verb|\otimes|&          
$\bigotimes$&       \verb|\bigotimes|&           
$\oslash$&          \verb|\oslash|\\       
$\odot$&            \verb|\odot|&               
$\bigodot$&        \verb|\bigodot|&            
$\bigcirc$&         \verb|\bigcirc|\\             
$\amalg$&           \verb|\amalg|&              
$\leq$&             \verb|\leq|&             
$\prec$&           \verb|\prec|\\               
$\preceq$&          \verb|\preceq|&              
$\ll$&              \verb|\ll|&             
$\subset$&          \verb|\subset|\\              
$\subseteq$&       \verb|\subseteq|&           
%$\sqsubset$&        \verb|\sqsubset|\\            
$\sqsubseteq$&      \verb|\sqsubseteq|&          
$\in$&              \verb|\in|\\       
$\vdash$&          \verb|\vdash|&              
$\geq$&             \verb|\geq|&                 
$\succ$&            \verb|\succ|\\                
$\succeq$&          \verb|\succeq|&              
$\gg$&             \verb|\gg|&           
$\supset$&          \verb|\supset|\\              
$\supseteq$&        \verb|\supseteq|&            
%$\sqsupset$&        \verb|\sqsupset|&            
$\sqsupseteq$&     \verb|\sqsupseteq|&         
$\ni$&              \verb|\ni|\\                  
$\dashv$&           \verb|\dashv|&               
$\equiv$&           \verb|\equiv|&               
$\sim$&            \verb|\sim|\\             
$\simeq$&           \verb|\simeq|&               
$\asymp$&           \verb|\asymp|&               
$\approx$&          \verb|\approx|\\              
$\cong$&           \verb|\cong|&               
$\neq$&             \verb|\neq|&              
$\doteq$&           \verb|\doteq|\\               
$\propto$&          \verb|\propto|&              
$\models$&         \verb|\models|&             
$\perp$&            \verb|\perp|\\               
$\mid$&             \verb|\mid|&                 
$\parallel$&       \verb|\parallel|&           
$\bowtie$&         \verb|\bowtie|\\           
%$\Join$&            \verb|\Join|&                
$\smile$&           \verb|\smile|&               
$\frown$&           \verb|\frown|&               
$\aleph$&           \verb|\aleph|\\               
$\hbar$&           \verb|\hbar|&             
$\imath$&           \verb|\imath|&               
$\jmath$&           \verb|\jmath|\\               
$\ell$&             \verb|\ell|&               
$\wp$&             \verb|\wp|&                 
$\Re$&              \verb|\Re|\\                
$\Im$&              \verb|\Im|&                  
%$\mho$&             \verb|\mho|&                 
$\prime$&          \verb|\prime|&              
$\empty$&           \verb|\empty|\\               
$\nabla$&           \verb|\nabla|&               
$\surd$&            \verb|\surd|&              
$\top$&            \verb|\top|\\              
$\bot$&             \verb|\bot|&                 
$\|$&               \verb?\|?&
$\angle$&           \verb|\angle|\\               
$\forall$&         \verb|\forall|&             
$\exists$&          \verb|\exists|&              
$\neg$&             \verb|\neg|\\              
$\flat$&            \verb|\flat|&                
$\natural$&        \verb|\natural|&            
$\sharp$&           \verb|\sharp|\\               
$\backslash$&        \verb|\backslash|&            
$\partial$&         \verb|\partial|&          
$\infty$&          \verb|\infty|\\              
%$\Box$&             \verb|\Box|&           
%$\Diamond$&         \verb|\Diamond|&             
$\triangle$&        \verb|\triangle|&            
$\sum$&            \verb|\sum|&              
$\prod$&            \verb|\prod|\\            
$\coprod$&          \verb|\coprod|&           
$\int$&             \verb|\int|&                 
$\oint$&           \verb|\oint|\\               
\end{longtable}

\subsubsection{Arrows}
\begin{longtable}{llllll}
$\leftarrow$&      \verb|\leftarrow|&
$\Leftarrow$&       \verb|\Leftarrow|&          
$\rightarrow$&      \verb|\rightarrow|\\          
$\Rightarrow$&      \verb|\Rightarrow|&          
$\leftrightarrow$& \verb|\leftrightarrow|&     
$\Leftrightarrow$&  \verb|\Leftrightarrow|\\      
$\mapsto$&          \verb|\mapsto|&              
$\hookleftarrow$&   \verb|\hookleftarrow|&      
$\leftharpoonup$&  \verb|\leftharpoonup|\\     
$\leftharpoondown$& \verb|\leftharpoondown|&     
$\rightleftharpoons$& \verb|\rightleftharpoons|& 
$\longleftarrow$&   \verb|\longleftarrow|\\  
$\Longleftarrow$&  \verb|\Longleftarrow|&      
$\longrightarrow$&  \verb|\longrightarrow|&      
$\Longrightarrow$&  \verb|\Longrightarrow|\\      
$\longleftrightarrow$& \verb|\longleftrightarrow|&
$\Longleftrightarrow$& \verb|\Longleftrightarrow|&
$\longmapsto$&      \verb|\longmapsto|\\          
$\hookrightarrow$&  \verb|\hookrightarrow|&      
$\rightharpoonup$&  \verb|\rightharpoonup|&     
$\rightharpoondown$& \verb|\rightharpoondown|\\ 
%$\leadsto$&         \verb|\leadsto|&             
$\uparrow$&         \verb|\uparrow|&             
$\Uparrow$&         \verb|\Uparrow|\\             
$\downarrow$&      \verb|\downarrow|&          
$\Downarrow$&       \verb|\Downarrow|&           
$\updownarrow$&     \verb|\updownarrow|\\         
$\nearrow$&         \verb|\nearrow|&        
$\searrow$&        \verb|\searrow|&            
$\swarrow$&         \verb|\swarrow|\\             
$\nwarrow$&         \verb|\nwarrow|&  & & & \\
\end{longtable}            


\subsubsection{Calligraphic}
These characters are available if you use the \verb|\cal| control
sequence.
\begin{verbatim} 
${\cal A B C D E F G H I J K L M N O P Q R S T U V W X Y Z}$
\end{verbatim}
gives
${\cal A B C D E F G H I J K L M N O P Q R S T U V W X Y Z}$

\subsubsection{Character Modifiers}
\begin{longtable}{llll}
\verb|\hat{e}|&            $\hat{e}$&            
\verb|\widehat{easy}|&      $\widehat{easy}$ \\     
\verb|\tilde{e}|&          $\tilde{e}$&          
\verb|\widetilde{easy}|&   $\widetilde{easy}$\\   
\verb|\check{e}|&          $\check{e}$&          
\verb|\breve{e}|&          $\breve{e}$\\         
\verb|\acute{e}|&          $\acute{e}$&          
\verb|\grave{e}|&          $\grave{e}$\\          
\verb|\bar{e}|&            $\bar{e}$&            
\verb|\vec{e}|&            $\vec{e}$\\            
\verb|\dot{e}|&            $\dot{e}$&            
\verb|\ddot{e}|&           $\ddot{e}$\\           
\verb|\not e| &            $\not e$ & & \\
\end{longtable}

\vspace{.5cm}
Note that the wide versions of \texttt{hat} and \texttt{tilde} cannot produce
very wide alternatives. The `\verb|\not|' operator hasn't properly 
cut the following letter. The \emph{Fine Tuning} section on page 
\pageref{FINE_TUNING}
describes how to adjust this.

 If you want to place one character 
above another, you can use \verb|\stackrel|, which prints 
its first argument in small type immediately above the second
\begin{verbatim}
$ a \stackrel{def}{=} b + c $
\end{verbatim}
gives
$ a \stackrel{def}{=} b + c $

See the \emph{Macros} section for how to stack characters using \texttt{atop}.

\subsubsection{Common functions}
In a maths environment, \LaTeX\ assumes that variables will 
have single-character names. Function names require 
special treatment.
The advantage of using the following control sequences for
common functions is that the text will not
be put in math italic and subscripts/superscripts will be made into limits
where appropriate.

\begin{tabular}{lllllll}
\verb|\arccos|& \verb|\arcsin|& \verb|\arctan|& \verb|\arg|& \verb|\cos| & 
\verb|\cosh|& \verb|\cot|\\
\verb|\coth|& \verb|\csc|& \verb|\deg|& 
\verb|\det|& \verb|\dim|& \verb|\exp|& \verb|\gcd|\\
\verb|\hom| & \verb|\inf|& \verb|\ker|& \verb|\lg|& \verb|\lim|&
\verb|\liminf|& \verb|\ln|\\
\verb|\log| & \verb|\max|& \verb|\min|& \verb|\Pr|&
\verb|sec|& \verb|\sin|& \verb|\sinh|\\
 \verb|\sup|& \verb|\tan| & \verb|\tanh| & & &  &\\    
\end{tabular}



\subsection{Subscripts and superscripts}
These are introduced by the `\verb|^|' and `\verb|_|' characters.
Depending on the base character and the current style, the sub- or 
superscripts may go to the right of or directly above/below the main 
character.
With letters it goes to the right.
\begin{verbatim}
$F_2^3$
\end{verbatim}
produces `$F_2^3$'. Note that the sub- and superscripts aren't in line.
To make them so, you can add an invisible character after the `F'. 
\verb|$F{}_2^3$| produces $F{}_2^3$.

With $\sum$ the default behaviour is different in display and
text styles.

\begin{verbatim}
$\sum_{i=0}^2 $ 
\end{verbatim}
produces $\sum_{i=0}^2 $ (text style) but 
\begin{verbatim}
\[\sum_{i=0}^2 \] 
\end{verbatim}
produces (in display style) \[\sum_{i=0}^2 \] 

This default behaviour can be overridden, if you really need
to. For example in text mode,
\begin{verbatim}
$\sum\limits_{i=0}^2$ 
\end{verbatim}
produces $\sum\limits_{i=0}^2$

\subsection{Overlining, underlining and bold characters}
\begin{verbatim}
$\underline{one} \overline{two}$ 
\end{verbatim}
produces $\underline{one} \overline{two}$. This is not a 
useful facility if it's used more than once on a line. The lines are
produced so that they don't quite overlap the text; lines over or 
under different words won't in general be at the same height. 

To be able to reproduce bold maths, it's best to use the \texttt{bm}
package. \verb|$E = \bm{mc^2}$| produces $E = \bm{mc^2}$.

Alternatively, you can use \verb|\mathbf{}| to create bold characters -
\verb|$\mathbf{F}_2^3$| produces $\mathbf{F}_2^3$. 
or you can use the 
following idea
\begin{verbatim}
\usepackage{amsbsy} % This loads amstext too
\begin{document}
$\omega + \boldsymbol{\omega}$

% Use the following if whole expressions need to be in bold
{\boldmath $\omega $}
\end{document}
\end{verbatim}

\subsection{Roots}
\begin{verbatim}
$\sqrt{4} + \sqrt[3]{x + y}$
\end{verbatim}
gives $\sqrt{4} + \sqrt[3]{x + y}$.

\subsection{Fractions}
Three constructions for putting expressions above others are
\begin{description}
\item[frac:-]
\verb|$\frac{1}{(x + 3)}$| produces $\frac{1}{(x + 3)}$.
\item[choose:-]
\verb|${n + 1 \choose 3}$| produces ${n + 1 \choose 3}$.
\item[atop:-]
\verb|${x \atop y}$| produces ${x \atop y}$.  
\end{description}

These constructions can be used with ones described earlier. 
\emph{E.g.},
\begin{verbatim}
\[ \sum_{-1\le i \le 1 \atop 0 < j < \infty} f(i,j)\]
\end{verbatim}
gives \[ \sum_{-1\le i \le 1 \atop 0 < j < \infty} f(i,j) \]


\subsection{Delimiters}

\begin{center}
\begin{tabular}{|l|l||l|l|}
\hline
\emph{these} & are made by these & \emph{and these} & are made by these\\
\hline
$( $ & \verb|( | &
$) $ &\verb|) | \\
$[ $ & \verb|[ | &
$] $ & \verb|] |\\
$\{$ & \verb|\{| &
$\} $ & \verb|\} |\\
$\lfloor $ &\verb|\lfloor | &
$\rfloor $ &\verb|\rfloor | \\
$\lceil $ &\verb|\lceil | &
$\rceil $ &\verb|\rceil | \\
$\langle $ &\verb|\langle | &
$\rangle $ & \verb|\rangle |\\
$/ $ & \verb|/ | &
$\backslash $ &\verb|\backslash | \\
$| $ & \verb?|? &
$\| $ & \verb?\|?\\
$\uparrow $ &\verb|\uparrow| &
$\Uparrow $ & \verb|\Uparrow|\\
$\downarrow $ &\verb|\downarrow| &
$\Downarrow $ & \verb|\Downarrow|\\
$\updownarrow $ &\verb|\updownarrow| &
$\Updownarrow $ &\verb|\Updownarrow| \\
\hline
\end{tabular}
\end{center}

This table shows the standard sizes. To get bigger sizes, use these prefices

\begin{tabular}{|l|l|l|}
\hline
(for left delimiters) & (for right delimiters) & magnification\\
\hline
\verb|\bigl|  & \verb|\bigr|  & a bit bigger, but won't overlap lines\\
\verb|\Bigl|  & \verb|\Bigr|  & 150\%  times  \verb|big|\\
\verb|\biggl| & \verb|\biggr| & 200\% times  \verb|big|\\
\verb|\Biggl| & \verb|\Biggr| & 250\% times  \verb|big|\\
\hline
\end{tabular}

For example,
\begin{verbatim}
$\Biggl\{2\Bigl(x(3+y)\Bigr)\Biggr\}$
\end{verbatim}
gives $\Biggl\{2\Bigl(x(3+y)\Bigr)\Biggr\}$. If you're not using the default
text size these commands might not work correctly. In that case try the
\texttt{exscale} package. 

It's preferable to let \LaTeX\ choose the delimiter size for you by using 
\verb|\left| and \verb|\right|. These will produce delimiters just
big enough for the formulae inbetween.
\begin{verbatim}
$\left( \frac{(x+iy)}{\{\int x\}} \right)$
\end{verbatim}
gives $\left( (x+iy) \over \{\int x\} \right)$

The left and right delimiters needn't be the same type. It's sometimes
useful to make one of them invisible

\begin{verbatim}
\[ z = \left\{
              \begin{array}{ll}
                   1 & (x>0)\\
                   0 & (x<0)
              \end{array}
       \right. 
\]
\end{verbatim}
produces

\[ z = \left\{
              \begin{array}{ll}
                   1 & (x>0)\\
                   0 & (x<0)
              \end{array}
       \right. 
\]



Over- and underbracing works too.
\begin{verbatim}
$\overbrace{\alpha \ldots \omega}^{\mbox{greek}} 
 \underbrace{a \ldots z}_{\mbox{english}}$
\end{verbatim}
produces 
$\overbrace{\alpha \ldots \omega}^{\mbox{greek}} 
 \underbrace{a \ldots z}_{\mbox {english}}$.
The use of \verb|\mbox| stops the text appearing in math italic.

\subsection{Numbering and labelling}
Numbering happening automatically when you display equations. If you 
\emph{don't} want an equation numbered, use \verb|\nonumber| beside the
equation. Equation numbers appear to the right of the maths by default.
To make them appear on the left use the \texttt{leqno} class option
(i.e., use \verb|\documentclass[leqno,....]{....}|).


Use \verb|\label{}| to label an equation (or figure, section etc) in 
order to reference from elsewhere. 

\begin{verbatim}
\begin{equation}
W_{\bf S}(t,\omega) = \int\limits_{-\infty}^{\infty} {
  {\cal R}_{\bf S}(t,\tau) e^{-j\omega\tau} \,d \tau }
\label{LABELLING}
\end{equation}
\end{verbatim}
\begin{equation}
W_{\bf S}(t,\omega) = \int\limits_{-\infty}^{\infty} {
  {\cal R}_{\bf S}(t,\tau) e^{-j\omega\tau} \,d \tau }
\label{LABELLING}
\end{equation}

Now the following text
\begin{verbatim}
 refers back to equation \ref{LABELLING}
\end{verbatim}
 refers back to equation \ref{LABELLING} by number, and
\begin{verbatim} 
 refers back to the equation on page \pageref{LABELLING}
\end{verbatim}
 refers back to the equation on page \pageref{LABELLING}.

A file will have to be \LaTeX 'ed twice before the references,
both forwards and backwards, will be correctly produced. 


\subsection{Matrices}
The \texttt{array} environment is like \LaTeX 's \texttt{tabular} environment
except that each element is in \texttt{math} mode. The number and alignment
of columns is controlled by the arguments - use \texttt{l, c} 
or \texttt{r} to represent each column with either left, center or right 
alignment. The default font style 
used is \verb|\textstyle| but you can override this by changing the \verb|\displaystyle|.


\begin{verbatim}
\begin{math}
\begin{array}{clrr} %
      a+b+c & uv & x-y & 27 \\
       x+y  & w  & +z  & 363 
\end{array}
\end{math}
\end{verbatim}
produces
\begin{math}
\begin{array}{clrr}
      a+b+c & uv & x-y & 27 \\
       x+y  & w  & +z  & 363 
\end{array}
\end{math}

The rows are arranged so that their centres are aligned. You 
can align their tops or bottoms instead by using a further argument
when you create the array.
\begin{verbatim}
\begin{array}{clrr}[t]
\end{verbatim}
would produce top-aligned lines, and \verb|`[b]'| would produce
bottom-aligned ones. The \emph{Delimiters} section of this document
shows how to bracket matrices.

\TeX\ has a few maths facilities not mentioned in the \LaTeX\ book.
The following \TeX\ construction might be useful.
\begin{verbatim}
\begin{math}
\bordermatrix{&a_1&a_2&...&a_n\cr
          b_1 & 1.2  & 3.3  & 5.1  & 2.8  \cr
          c_1 & 4.7  & 7.8  & 2.4  & 1.9  \cr
          ... & ...  & ...  & ...  & ...  \cr
          z_1 & 8.0  & 9.9  & 0.9  & 9.99  \cr}
\end{math}
\end{verbatim}

\begin{math}
\bordermatrix{&a_1&a_2&...&a_n\cr
          b_1 & 1.2  & 3.3  & 5.1  & 2.8   \cr
          c_1 & 4.7  & 7.8  & 2.4  &  1.9    \cr
          ... & ...  & ...  & ...  & ...   \cr
          z_1 & 8.0  & 9.9  & 0.9  &  9.99  \cr}
\end{math}


\subsection{Macros}
These aid readability, save on repetitive typing and offer ways of
producing stylistic variations on standard \LaTeX\ formats.
\begin{verbatim}

\def\bydefn{\stackrel{def}{=}}
\def\convf{\hbox{\space \raise-2mm\hbox{$\textstyle  
    \bigotimes \atop \scriptstyle \omega$} \space}}
\end{verbatim}

produce $\bydefn$ and $\convf$ when \verb|$\bydefn$| and \verb|$\convf$|
are typed.

\subsection{Packages}
The following packages may be of help 
\begin{itemize}
\item \htmladdnormallinkfoot{easybmat}{http://www-h.eng.cam.ac.uk/help/tpl/textprocessing/easybmat.dvi} - easy block matrices
\item \htmladdnormallinkfoot{easyeqn}{http://www-h.eng.cam.ac.uk/help/tpl/textprocessing/easyeqn.dvi} - easy equations.
\item \htmladdnormallinkfoot{easymat}{http://www-h.eng.cam.ac.uk/help/tpl/textprocessing/easymat.dvi} - easy matrices
\item \htmladdnormallinkfoot{easytable}{http://www-h.eng.cam.ac.uk/help/tpl/textprocessing/easytable.dvi} - easy tables
\item \htmladdnormallinkfoot{easyvector}{http://www-h.eng.cam.ac.uk/help/tpl/textprocessing/easyvector.dvi} - easy vectors
\item \htmladdnormallinkfoot{delarray}{http://www-h.eng.cam.ac.uk/help/tpl/textprocessing/delarray.dvi} - nested arrays
\item \htmladdnormallinkfoot{theorem}{http://www-h.eng.cam.ac.uk/help/tpl/textprocessing/theorem.dvi} - gives more choice in theorem layout
\item \texttt{subeqnarray} - Renumbering of sub-arrays in math-mode
\item \texttt{subeqn} - Different numbering sub-arrays
\end{itemize}


\subsection{Fine tuning}
\label{FINE_TUNING}
It's generally a good idea to keep punctuation outside math mode;
\LaTeX 's normal handling of spacing around punctuation is suspended 
during maths. Sometimes you might want to adjust the spacing in a 
formula (\emph{e.g.}, you might want to add space before \textit{dx}). Use these
symbols :-

\begin{tabular}{ll}
\verb|a\, b| & ($a\, b$) thin space \\
\verb|a\> b| & ($a\> b$) medium space \\
\verb|a\; b| & ($a\; b$) thick space \\
\verb|a\! b| & ($a\! b$) negative thin space \\
\end{tabular}

Long math expressions aren't broken automatically unless you use the
\htmladdnormallinkfoot{\texttt{breqn}}{http://www-h.eng.cam.ac.uk/help/tpl/textprocessing/breqndoc.dvi} package, which is still a little experimental.
In an \texttt{eqnarray} environment you may want to break a long line
manually. You can do this by putting
\begin{verbatim}
y & = & a + b \nonumber \\
  &   & + k
\end{verbatim}
to give
\begin{eqnarray}
y & = & a + b \nonumber \\
  &   & + k
\end{eqnarray}
but the spacing around the `\texttt{+}' on the 2nd line is wrong because
\LaTeX\ thinks it's a unary operator. You can fool \LaTeX\ into 
treating it as a binary operator by inserting a hidden character.
\begin{verbatim}
y & = & a + b \nonumber \\
  &   & \mbox{} + k
\end{verbatim}
gives
\begin{eqnarray}
y & = & a + b \nonumber \\
  &   & \mbox{} + k
\end{eqnarray}

You can use the \verb|\lefteqn| construction to format long expressions
so that continuation lines are differently indented.
\begin{verbatim}
\begin{eqnarray}
\lefteqn{x+ iy=}\\
 & & a + b + c + d + e + f + g + h + i + j + k +\nonumber\\
 & & l + m \nonumber
\end{eqnarray}
\end{verbatim}

\begin{eqnarray}
\lefteqn{x+ iy=}\\
 & & a + b + c + d + e + f + g + h + i + j + k +\nonumber \\
 & & l + m \nonumber
\end{eqnarray}

If you want more vertical spacing around a line you can create an 
invisible vertical "struct" in LaTeX. \verb|\rule[-.3cm]{0cm}{1cm}|
creates a box of width 0, height 1cm which starts .3cm below the usual 
line base. By adjusting these values you should be able to create as
much extra space below/above the maths as you like. ``$A \over B$  \rule[-.3cm]{0cm}{1cm}{and}'' is created by
\begin{verbatim}
$A \over B$  \rule[-.3cm]{0cm}{1cm}{and}
\end{verbatim}


\subsection{Maths and Postscript fonts}
It's easy to use a postscript font (like \usefont{T1}{phv}{m}{n}helvetica\usefont{T1}{ppl}{m}{n}) for the text of a 
\LaTeX{} document. What's harder is using the same font for maths.
An easy, reasonable option is to use the \texttt{mathptmx} package to 
put the maths into the postscript \textit{Times} and \textit{symbol} fonts where
possible. 

Alternatively, use
\begin{itemize}
\item the \texttt{mathpazo} package (loads Palatino as the text font family and a
mixture of the Pazo and CM fonts for math).

\item the \texttt{mathpple} package (loads Palatino as the text font family and a
mixture of artificially obliqued Euler fonts 
and CM fonts for math). 

\end{itemize}


Commercial and free alternatives are under development. 

\subsection{Matlab and LaTeX}
\htmladdnormallinkfoot{Matlab}{http://www-h.eng.cam.ac.uk/help/tpl/programs/matlab.html}
has some support for LaTeX production. For example
\begin{verbatim}
  latex('(sin(x)+2*x+3*x^2)/(5*x+6*x^2)','math.tex')
\end{verbatim}
puts the LaTeX representation of the expression into a file called
\texttt{math.tex}. Type ``\texttt{help latex}" inside 
matlab for details. 



\subsection{Examples}
\begin{itemize}
\item
\begin{verbatim}
\begin{equation}
\hat{\theta}_{w_i} = \hat{\theta}(s(t,{\cal U}_{w_i})).
\end{equation}
\end{verbatim}
gives
\begin{equation}
\hat{\theta}_{w_i} = \hat{\theta}(s(t,{\cal U}_{w_i})).
\end{equation}

\item
\begin{verbatim}
\begin{eqnarray}
{\cal M}^2(\hat{\theta},\theta) &=& E[(\hat{\theta} - \theta)^2]
\nonumber \\
{\cal M}^2(\hat{\theta},\theta) &=& {\rm var}^2(\hat{\theta}) + 
{\cal B}^2(\hat{\theta}).
\end{eqnarray}
\end{verbatim}
gives
\begin{eqnarray}
{\cal M}^2(\hat{\theta},\theta) &=& E[(\hat{\theta} - \theta)^2]
\nonumber \\
{\cal M}^2(\hat{\theta},\theta) &=& {\rm var}^2(\hat{\theta}) + 
{\cal B}^2(\hat{\theta}).
\end{eqnarray}

\item
\begin{verbatim}
\begin{equation}
\hat{W}_{s}(t,\omega;\phi) \bydefn 
\int\limits_{-\infty}^{\infty} 
{\hat{\cal R}_s(t,\tau;\psi) e^{-j\omega \tau}
\, d \tau }
\end{equation}
\end{verbatim}
gives
\begin{equation}
\hat{W}_{s}(t,\omega;\phi) \bydefn \int\limits_{-\infty}^{\infty} 
   {\hat{\cal R}_s(t,\tau;\psi)
e^{-j\omega \tau}
\, d \tau }
\end{equation}

\item
\begin{verbatim}
\begin{eqnarray}
{\cal B}(t,\omega) & \approx &
{1 \over 4\pi}
{\cal D}_t^2 W_{\bf S}(t, \omega)
{{{\scriptstyle \infty} \atop
{\displaystyle \int \! \int \!
}}\atop {\scriptstyle -\infty}}
t_1^2
\phi(t_1,\omega_1) \, dt_1 d\omega_1
\nonumber \\
&& +
{1 \over 4\pi}
{\cal D}_\omega^2 W_{\bf S}(t, \omega)
{{{\scriptstyle \infty} \atop
{\displaystyle \int \! \int \!
}}\atop {\scriptstyle -\infty}}
\omega_1^2
\phi(t_1,\omega_1) \, dt_1 \, d\omega_1.
\label{F4}
\end{eqnarray}
\end{verbatim}

gives
\begin{eqnarray}
{\cal B}(t,\omega) & \approx &
{1 \over 4\pi}
{\cal D}_t^2 W_{\bf S}(t, \omega)
{{{\scriptstyle \infty} \atop
{\displaystyle \int \! \int \!
}}\atop {\scriptstyle -\infty}}
t_1^2
\phi(t_1,\omega_1) \, dt_1 d\omega_1
\nonumber \\
&& +
{1 \over 4\pi}
{\cal D}_\omega^2 W_{\bf S}(t, \omega)
{{{\scriptstyle \infty} \atop
{\displaystyle \int \! \int \!
}}\atop {\scriptstyle -\infty}}
\omega_1^2
\phi(t_1,\omega_1) \, dt_1 \, d\omega_1.
\label{F4}
\end{eqnarray}

\item 
\begin{verbatim}
\newsavebox{\DERIVBOXZLM}
\savebox{\DERIVBOXZLM}[2.5em]{$\Longrightarrow\hspace{-1.5em}
\raisebox{.2ex}{*}
\hspace{-.7em}\raisebox{-.8ex}{\scriptsize lm}\hspace{.7em}$}
\newcommand{\Deriveszlm}{\usebox{\DERIVBOXZLM}}

\Deriveszlm
\end{verbatim}
gives 

\newsavebox{\DERIVBOXZLM}
\savebox{\DERIVBOXZLM}[2.5em]{$\Longrightarrow\hspace{-1.5em}
\raisebox{.2ex}{*}
\hspace{-.7em}\raisebox{-.8ex}{\scriptsize lm}\hspace{.7em}$}
\newcommand{\Deriveszlm}{\usebox{\DERIVBOXZLM}}

\Deriveszlm

\end{itemize}

\section{Graphics}
\LaTeX\ has a \texttt{picture} environment in which pictures can be drawn,
but you'll find graph paper handy. \texttt{xfig} can create code for the
\texttt{picture} environment but the resulting graphics still suffer several 
limitations: only certain slopes and circles can be reproduced. 
The best method presently available on Unix is to use \texttt{xfig} to produce
postscript files, which have no such limitations, but require a postscript
printer or equivalent. 

Whatever graphics you want to add, you should use the \texttt{figure}
environment so that \LaTeX{} can cope sensibly with situations where,
for example, you attempt to insert near the bottom of a page a
graphic that's half a page high. The \texttt{figure} environment
will \emph{float} the graphic to the top or bottom of the page, or
on the next page, with preferences that you can provide.

\begin{tabular}{>{\ttfamily}ll}
h & here \\
t & top of page \\
b & bottom of page \\
p & on a page with no text \\
\end{tabular}

Putting \texttt{!} as the first argument in the square brackets 
will encourage  \LaTeX\ to
do what you say, even if the result's sub-optimal. See the
online hints about
 \htmladdnormallinkfoot{floats in LaTeX}{http://www-h.eng.cam.ac.uk/help/tpl/textprocessing/float\_hint.html}
for further details.

\begin{verbatim}
\begin{figure}[htbp]
   \vspace{0.5in}
   \caption{0.5 inch of space}
\end{figure}
\end{verbatim}
\begin{figure}[htbp]
   \vspace{0.5in}
   \caption{0.5 inch of space}
\end{figure}

It's possible to have more than one graphic in a \emph{figure}. See the
example later on.

\subsection{Postscript}
\texttt{pdflatex} supports \texttt{JPEG}, \texttt{PNG}
and\texttt{PDF} images - but \emph{not} postscript. \texttt{latex} 
supports Postscript files as long as they have a proper 
bounding box comment; \emph{i.e.} \LaTeX{} requires full  
\htmladdnormallinkfoot{Encapsulated Postscript}{http://www-h.eng.cam.ac.uk/help/tpl/graphics/postscript.html} as 
produced by (for example) \texttt{xv} and \texttt{xfig} on Unix. 
If the file hasn't got a \texttt{BoundingBox} line near the top, you can 
use \texttt{ps2epsi} to generate one. Wherever the postscript file
comes from, simply use

\begin{verbatim}
\documentclass[dvips]{article}
\usepackage{graphicx}
\end{verbatim}
then include the postscript file using the following commands
\begin{verbatim}
\begin{figure}[htbp]
\includegraphics{yourfile.ps}
\end{figure}
\end{verbatim}

\LaTeX{} can cope with compressed postscript files too, but since latex
can't read the \texttt{BoundingBox} line from the compressed file, you
need to provide it. If your compressed file's called \texttt{yourfile.ps.gz}, 
copy the \texttt{BoundingBox} line into a file called
\texttt{yourfile.ps.bb}. Then the  following
works
\begin{verbatim}
\begin{figure}[htbp]
\includegraphics{yourfile.ps.gz}
\end{figure}
\end{verbatim}

Just about all of the following facilities use postscript. You'll need to
run \texttt{latex} to generate `\texttt{foo.dvi}'. This file can be viewed 
by the latest \texttt{xdvi} program, which can cope with embedded postscript.
Run \texttt{dvips -o foo.ps foo.dvi} 
to convert the resulting \texttt{DVI}/postscript file to pure postscript. 
This
will produce a file that can be previewed with \texttt{ghostview} or \texttt{gs}. On
the teaching system this file can be printed out using \texttt{plotview} or
\texttt{lp}. 

See the 
\htmladdnormallinkfoot{\emph{Creating and Printing graphics on PC, Mac, 
SUN and HP machines}}{http://www-h.eng.cam.ac.uk/help/tpl/graphics/graphics2/graphics2.html}
and \htmladdnormallinkfoot{\emph{Xfig}}{http://www-h.eng.cam.ac.uk/help/tpl/graphics/xfig/xfig.html} handouts for more details.

\subsubsection{\texttt{psfrag}: adding maths to postscript files}
Many packages that produce postscript output don't provide good maths
facilities. It's often easier to add the maths in later using the 
\texttt{psfrag} package.
This lets you replace text in a postscript file (produced with xfig,
matlab, etc) by a fragment of \LaTeX{}. For example, doing
\begin{verbatim}
\usepackage{psfrag}
...
\begin{figure}
\psfrag{MATHS}{$x^2$}
\includegraphics{foo.eps}
\end{figure}
\end{verbatim}
would display the file with \texttt{MATHS} replaced by $x^2$. See 
\htmladdnormallinkfoot{the online documentation}
{http://www-h.eng.cam.ac.uk/help/tpl/textprocessing/psfrag.ps}
for details.

\subsubsection{Postscript from PCs/Macs}
Modern applications should generate a conforming EPS file. Under windows
when you're printing to file, look at the PostScript properties (or
Advanced options), and choose (depending on the driver you have) either 
'Archive Format', 'Encapsulated PS', 'Optimize for Portability' or 'Page 
Independence'. People seem to have more luck with the free Adobe postscript 
driver than with the Microsoft one.

The resulting file should begin with \texttt{\%!PS}. If it doesn't it's not 
postscript. Remove any characters that are before   \texttt{\%!PS}, and
(to be on the safe side) remove anything after the final  \texttt{\%\%EOF} line.Platform-specific considerations do crop up. The EPS generated on
Macintoshes will use \verb|ASCII 13| line terminators, while Unix will
use \verb|ASCII 10| (and DOS will use both). If this causes trouble, 
use \texttt{emacs} to convert, or try (in Unix)
\begin{verbatim}
     tr "\015" "\012" <original.ps >new.ps
\end{verbatim}



\subsection{Scaling, rotation, clipping, wrap-around and shadows}

To scale, use some optional arguments
\begin{verbatim}
   \includegraphics[width=5cm,height=10cm]{yourfile.ps}
\end{verbatim}
would rescale the postscript so that it was 5cm wide and 10cm high.
To make the picture 5cm wide and scale the height in proportion use
\begin{verbatim}
   \includegraphics[width=5cm]{yourfile.ps}
\end{verbatim}

To rotate anticlockwise by the specified number of degrees, use
\begin{verbatim}
   \includegraphics[angle=150]{yourfile.ps}
\end{verbatim}

These options can be combined - note that order matters. 
The following examples demonstrate how to combine these features 
and how to use the \texttt{subfig} package to have more than one
graphic in a figure.

\begin{figure}[hbtp]
\centering
  \includegraphics[height=40mm]{/export/ghostfonts/tiger.eps} 
  \includegraphics[angle=120, height=20mm]{/export/ghostfonts/tiger.eps} 
\caption{Tigers}
\end{figure}

\begin{small}
\begin{verbatim}
\centering
\begin{figure}[hbtp]
  \includegraphics[height=40mm]{/export/ghostfonts/tiger.eps} 
  \includegraphics[angle=120, height=20mm]{/export/ghostfonts/tiger.eps} 
\caption{Tigers}
\end{figure}
\end{verbatim}

\begin{verbatim}
% remember to do \usepackage{subfig} at the top of the document! 
\begin{figure}[hbtp]
\centering
\subfloat[Small]% \quad on the next line adds spacing
{\includegraphics[height=30mm]{/export/ghostfonts/crest.eps}}\quad
\subfloat[Medium]
{\includegraphics[width=40mm]{CUni3.eps}}\quad
\subfloat[Large]
{\includegraphics[height=50mm]{/export/ghostfonts/crest.eps}} 
\caption{3 crests}
\end{figure}
\end{verbatim}
\end{small}

\begin{figure}[hbtp]
\centering
\subfloat[Small]
{\includegraphics[height=30mm]{/export/ghostfonts/crest.eps}}\quad
\subfloat[Medium]
{\includegraphics[width=40mm]{CUni3.eps}}\quad
\subfloat[Large]
{\includegraphics[height=50mm]{/export/ghostfonts/crest.eps}} 
\caption{3 crests}
\end{figure}

To clip the postscript image use the \texttt{viewpoint} argument. 
The following fragment would display only part of the image.
The viewport coordinates are in the same units as the bounding box.

\begin{verbatim}
\begin{figure}[htbp]
\includegraphics[viewport=200 400 400 600,width=5cm,clip]%
{CUniv3.eps}
\end{figure}
\end{verbatim}


\begin{verbatim}
% Use the floatflt package
\begin{floatingfigure}[l]{4cm}
\includegraphics[width=2cm]{/export/ghostfonts/crest.eps}  
\caption{Using floatingfigure}
\end{floatingfigure}
\end{verbatim}

\begin{floatingfigure}[l]{4cm}
\includegraphics[width=2cm]{/export/ghostfonts/crest.eps}  
\caption{Using floatingfigure}
\end{floatingfigure}
The 
\htmladdnormallinkfoot{\texttt{floatflt}}{http://www-h.eng.cam.ac.uk/help/tpl/textprocessing/floatflt.dvi}
package lets you insert a graphic and have the
text wrap around it. You can provide 2 arguments to the \texttt{floatingfigure}
command: the first (\texttt{l} or
 \texttt{r}) selects whether you want the graphic to be on the left
or right of the page. The 2nd argument gives the width of the graphic.
Not all text will flow perfectly around (for example, \texttt{verbatim}
text fails, as illustrated below) so check the final output carefully.

Using the \texttt{fancybox} package gives you access to 
\verb|\shadowbox|, \verb|\ovalbox|,  \verb|\Ovalbox| and
\verb|\doublebox| commands, which can be used with text or
with graphics. For example, \verb|\shadowbox{shadow package}| produces
\shadowbox{shadow package} and 
\begin{verbatim}
\ovalbox{\includegraphics[height=10mm]{CUni3.eps}}
\end{verbatim}
 produces
\ovalbox{\includegraphics[height=10mm]{CUni3.eps}}.
Unfortunately, the \texttt{fancybox} package as supplied suppresses the
table of contents.  The locally produced \texttt{contentsfancybox} solves 
this, but may introduce graphics problems.

\subsection{GIF and jpeg files}
CUED users can include a file called \texttt{keyboard.gif}, for example,
by doing \texttt{gif2ps -b keyboard.gif} (as long as you don't change the
\texttt{GIF} file you need only do this once) and then including the 
\texttt{GIF} file as you would a postscript one.

For JPEG files run "\texttt{jpeg2ps -h} \textit{file.jpg} $>$ 
\textit{file.eps}" then include the postscript file in the usual way. 
The resulting \texttt{eps} file will be little bigger than the original
file. Alternatively, if you put the 'Bounding Box' line from the  \texttt{eps} file and put it in \textit{file.jpg.bb} you can include JPEG files in
the same way that you do postscript files.
\end{document}
